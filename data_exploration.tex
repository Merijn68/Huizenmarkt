% Options for packages loaded elsewhere
\PassOptionsToPackage{unicode}{hyperref}
\PassOptionsToPackage{hyphens}{url}
%
\documentclass[
]{article}
\usepackage{lmodern}
\usepackage{amssymb,amsmath}
\usepackage{ifxetex,ifluatex}
\ifnum 0\ifxetex 1\fi\ifluatex 1\fi=0 % if pdftex
  \usepackage[T1]{fontenc}
  \usepackage[utf8]{inputenc}
  \usepackage{textcomp} % provide euro and other symbols
\else % if luatex or xetex
  \usepackage{unicode-math}
  \defaultfontfeatures{Scale=MatchLowercase}
  \defaultfontfeatures[\rmfamily]{Ligatures=TeX,Scale=1}
\fi
% Use upquote if available, for straight quotes in verbatim environments
\IfFileExists{upquote.sty}{\usepackage{upquote}}{}
\IfFileExists{microtype.sty}{% use microtype if available
  \usepackage[]{microtype}
  \UseMicrotypeSet[protrusion]{basicmath} % disable protrusion for tt fonts
}{}
\makeatletter
\@ifundefined{KOMAClassName}{% if non-KOMA class
  \IfFileExists{parskip.sty}{%
    \usepackage{parskip}
  }{% else
    \setlength{\parindent}{0pt}
    \setlength{\parskip}{6pt plus 2pt minus 1pt}}
}{% if KOMA class
  \KOMAoptions{parskip=half}}
\makeatother
\usepackage{xcolor}
\IfFileExists{xurl.sty}{\usepackage{xurl}}{} % add URL line breaks if available
\IfFileExists{bookmark.sty}{\usepackage{bookmark}}{\usepackage{hyperref}}
\hypersetup{
  pdftitle={Ontwikkelingen huizenmarkt},
  pdfauthor={Merijn van Miltenburg},
  hidelinks,
  pdfcreator={LaTeX via pandoc}}
\urlstyle{same} % disable monospaced font for URLs
\usepackage[margin=1in]{geometry}
\usepackage{color}
\usepackage{fancyvrb}
\newcommand{\VerbBar}{|}
\newcommand{\VERB}{\Verb[commandchars=\\\{\}]}
\DefineVerbatimEnvironment{Highlighting}{Verbatim}{commandchars=\\\{\}}
% Add ',fontsize=\small' for more characters per line
\usepackage{framed}
\definecolor{shadecolor}{RGB}{248,248,248}
\newenvironment{Shaded}{\begin{snugshade}}{\end{snugshade}}
\newcommand{\AlertTok}[1]{\textcolor[rgb]{0.94,0.16,0.16}{#1}}
\newcommand{\AnnotationTok}[1]{\textcolor[rgb]{0.56,0.35,0.01}{\textbf{\textit{#1}}}}
\newcommand{\AttributeTok}[1]{\textcolor[rgb]{0.77,0.63,0.00}{#1}}
\newcommand{\BaseNTok}[1]{\textcolor[rgb]{0.00,0.00,0.81}{#1}}
\newcommand{\BuiltInTok}[1]{#1}
\newcommand{\CharTok}[1]{\textcolor[rgb]{0.31,0.60,0.02}{#1}}
\newcommand{\CommentTok}[1]{\textcolor[rgb]{0.56,0.35,0.01}{\textit{#1}}}
\newcommand{\CommentVarTok}[1]{\textcolor[rgb]{0.56,0.35,0.01}{\textbf{\textit{#1}}}}
\newcommand{\ConstantTok}[1]{\textcolor[rgb]{0.00,0.00,0.00}{#1}}
\newcommand{\ControlFlowTok}[1]{\textcolor[rgb]{0.13,0.29,0.53}{\textbf{#1}}}
\newcommand{\DataTypeTok}[1]{\textcolor[rgb]{0.13,0.29,0.53}{#1}}
\newcommand{\DecValTok}[1]{\textcolor[rgb]{0.00,0.00,0.81}{#1}}
\newcommand{\DocumentationTok}[1]{\textcolor[rgb]{0.56,0.35,0.01}{\textbf{\textit{#1}}}}
\newcommand{\ErrorTok}[1]{\textcolor[rgb]{0.64,0.00,0.00}{\textbf{#1}}}
\newcommand{\ExtensionTok}[1]{#1}
\newcommand{\FloatTok}[1]{\textcolor[rgb]{0.00,0.00,0.81}{#1}}
\newcommand{\FunctionTok}[1]{\textcolor[rgb]{0.00,0.00,0.00}{#1}}
\newcommand{\ImportTok}[1]{#1}
\newcommand{\InformationTok}[1]{\textcolor[rgb]{0.56,0.35,0.01}{\textbf{\textit{#1}}}}
\newcommand{\KeywordTok}[1]{\textcolor[rgb]{0.13,0.29,0.53}{\textbf{#1}}}
\newcommand{\NormalTok}[1]{#1}
\newcommand{\OperatorTok}[1]{\textcolor[rgb]{0.81,0.36,0.00}{\textbf{#1}}}
\newcommand{\OtherTok}[1]{\textcolor[rgb]{0.56,0.35,0.01}{#1}}
\newcommand{\PreprocessorTok}[1]{\textcolor[rgb]{0.56,0.35,0.01}{\textit{#1}}}
\newcommand{\RegionMarkerTok}[1]{#1}
\newcommand{\SpecialCharTok}[1]{\textcolor[rgb]{0.00,0.00,0.00}{#1}}
\newcommand{\SpecialStringTok}[1]{\textcolor[rgb]{0.31,0.60,0.02}{#1}}
\newcommand{\StringTok}[1]{\textcolor[rgb]{0.31,0.60,0.02}{#1}}
\newcommand{\VariableTok}[1]{\textcolor[rgb]{0.00,0.00,0.00}{#1}}
\newcommand{\VerbatimStringTok}[1]{\textcolor[rgb]{0.31,0.60,0.02}{#1}}
\newcommand{\WarningTok}[1]{\textcolor[rgb]{0.56,0.35,0.01}{\textbf{\textit{#1}}}}
\usepackage{graphicx,grffile}
\makeatletter
\def\maxwidth{\ifdim\Gin@nat@width>\linewidth\linewidth\else\Gin@nat@width\fi}
\def\maxheight{\ifdim\Gin@nat@height>\textheight\textheight\else\Gin@nat@height\fi}
\makeatother
% Scale images if necessary, so that they will not overflow the page
% margins by default, and it is still possible to overwrite the defaults
% using explicit options in \includegraphics[width, height, ...]{}
\setkeys{Gin}{width=\maxwidth,height=\maxheight,keepaspectratio}
% Set default figure placement to htbp
\makeatletter
\def\fps@figure{htbp}
\makeatother
\setlength{\emergencystretch}{3em} % prevent overfull lines
\providecommand{\tightlist}{%
  \setlength{\itemsep}{0pt}\setlength{\parskip}{0pt}}
\setcounter{secnumdepth}{5}
\usepackage{booktabs}
\usepackage{longtable}
\usepackage{array}
\usepackage{multirow}
\usepackage{wrapfig}
\usepackage{float}
\usepackage{colortbl}
\usepackage{pdflscape}
\usepackage{tabu}
\usepackage{threeparttable}
\usepackage{threeparttablex}
\usepackage[normalem]{ulem}
\usepackage{makecell}
\usepackage{xcolor}

\title{Ontwikkelingen huizenmarkt}
\usepackage{etoolbox}
\makeatletter
\providecommand{\subtitle}[1]{% add subtitle to \maketitle
  \apptocmd{\@title}{\par {\large #1 \par}}{}{}
}
\makeatother
\subtitle{Huizenprijzen en transactievolumes in Nederland}
\author{Merijn van Miltenburg}
\date{08/04/2021}

\begin{document}
\maketitle

{
\setcounter{tocdepth}{2}
\tableofcontents
}
\hypertarget{inleiding}{%
\section{Inleiding}\label{inleiding}}

De volksbank is het moederbedrijf van ASN, SNS, BLG en Regiobank. Als de
vierde bank van Nederland, financiert de volksbank hypotheken, beheert
ze spaargeld en biedt de bank klanten een betaalrekening. Binnen de
afdeling Financial Markets van de Volksbank, hebben de macro-economen
van de Volksbank als taak de toestand in de economie te monitoren en toe
te lichten aan directie en beleidsmakers. Een belangrijk onderdeel
daarvan heeft betrekking met de ontwikkeling van de hypotheekmarkt in
Nederland. De huizenmarkt is een belangrijke factor in de
beleidsontwikkeling van de bank, met betrekking tot de
hypotheekproductie.

Voor deze analyse is de PPDAC cyclus ( Problem, Plan, Data, Analysis,
and Conclusions) gevolgd. Een methode voor het uitvoeren van statistisch
onderzoek (MacKay \& Oldford, 2000). De opbouw van dit verslag volgt
deze cyclus. In de volgende paragraaf wordt eerst de probleemstelling
nader toegelicht. Vervolgens is in de planningsfase gekeken wat de
belangrijkste indicatoren zijn om de vraagstelling te beantwoorden. De
data die nodig is voor dit onderzoek is verzameld en vervolgens
geanalyseerd. Op basis daarvan worden tot slot een aantal conclusies
getrokken. Deze cyclus is een aantal keer doorlopen - op basis van de
uitkomsten. De fasering van deze cyclus komt terug in de
probleemstelling en de hoofdstukindeling van deze studie.

\hypertarget{probleemstelling}{%
\section{Probleemstelling}\label{probleemstelling}}

Uit het overleg met de macro-economen is de vraag naar voren gekomen of
een beter voorspellend model kan worden ontwikkeld voor de huizenmarkt
in Nederland.

Is het mogelijk om modellen te ontwikkelen die de huizenprijs op
middellange termijn voorspelt en het aantal transacties op de
huizenmarkt op middellange termijn voorspelt?"

Vanuit de Volksbank is in het verleden al eerder statistisch onderzoek
gedaan naar deze verbanden. Voor de huizenprijzen is eerder gekeken naar
een verband tussen huizenprijzen en de werkgelegenheid, lonen, BBP
(Bruto Binnenlands Product) en rente. In deze exploratie wordt gekeken
naar mogelijke factoren die effect hebben op de marktprijs. De hier
genoemde factoren zullen hierbij worden meegenomen.

Welke factoren spelen een belangrijke rol in het bepalen van de
huizenprijzen en transactie volumes op een termijn van 2 tot 5 jaar?

Voor beantwoording van deze vraag wordt eerst gekeken in hoeverre de
huizenprijzen worden bepaald door wijzigingen in vraag en aanbod. Daarna
wordt verder ingegaan op de factoren die op langere termijn de
huizenprijzen beïnvloeden en wordt gekeken of hier voorspellende waarde
in zit.

Vervolgens wordt gekeken naar de transactievolumes. Welke factoren
spelen een belangrijke rol in het bepalen van de transactievolumes op
een termijn van 2 tot 5 jaar? Voor het aantal transacties op de lange
termijn wordt een verband verondersteld met het aantal huishoudens. Op
kortere termijn spelen economische groei, rentestanden en ook
verandering van regelgeving een rol. In welke mate spelen
seizoensinvloeden hierbij een rol?

Aanvullend zijn er nog een aantal vragen geformuleerd waar mogelijk door
middel van exploratie iets meer over kan worden gezegd:

\begin{enumerate}
\def\labelenumi{\arabic{enumi}.}
\item
  Nederland heeft een van de hoogste persoonlijke schulden
  (hypotheeklast) in de wereld. Wat zijn de gevolgen voor de
  hypotheekmarkt van het gewijzigde beleid? Heeft dit effect op de
  huizenprijzen?
\item
  Hoe ontwikkeld de krapte op de huizenmarkt zich, op termijn? Momenteel
  zijn er ambitieuze plannen om ergens tussen de 75.000 en 100.000
  woningen per jaar te gaan bouwen. Het aanbod aan huizen wordt dus
  groter - terwijl op termijn verwacht wordt dat de bevolking weer gaat
  krimpen. Kunnen we hier een voorspelling van maken?
\item
  Hoe zit het met de bevolkingsgroei? Welk deel van de aanwas in
  Nederland komt door migratie? Hoe zit dat regionaal? Is er nog sprake
  van verstedelijking na corona? Stroomt het platteland leeg - en de
  randstad vol? Welk effect heeft dit op de huizenprijzen?
\item
  Zien we demografisch verdeeld verschillen waar mensen willen wonen?
  Verhuizen gezinnen met kinderen naar het platteland, en ouderen weer
  naar de stad? Het aantal huishoudens dat uit 1 persoon bestaat is erg
  gegroeid in de loop der tijd - en zal (denk ik) nog wel doorstijgen
  met de vergrijzing. Betekent dit dat er meer mensen naar de stad
  willen verhuizen?
\item
  Hoe afhankelijk is de huizenmarkt van de stand van de economie? Gaan
  huizenprijzen alleen omhoog tijdens de groeifase van de economie?
\end{enumerate}

\hypertarget{plan}{%
\section{Plan}\label{plan}}

Om een begin te maken met het beantwoorden van deze vragen zijn een
aantal gesprekken gevoerd met specialisten bij de Volksbank. Aanvullend
is een korte literatuurstudie uitgevoerd naar de factoren die op lange
termijn de prijsstelling beïnvloeden. Het CBS publiceert vrij veel over
de prijsontwikkeling van bestaande woningen, nieuwe woningen en de stand
van de economie (CBS, 2016), (CBS, 2020), (``Monitor koopwoningmarkt,''
2020), (Groot, Vogt, Van der Wiel, \& Van Dijk, 2018). De Socrates
modellen sluiten goed aan bij deze vraagstelling (Kenneth Gopal, Gerard
van Leeuwen, David Omtzigt, 2020).

Vraag en aanbod worden bepaald door:

\begin{column}{0.48\textwidth}

Vraag woningen:

\begin{itemize}
\tightlist
\item
  Aantal huishoudens
\item
  Demografie (omvang en samenstelling huishoudens)
\item
  (Arbeids-)migratie
\end{itemize}


\end{column}

\begin{column}{0.04\textwidth}
~

\end{column}

\begin{column}{0.48\textwidth}

Aanbod woningen:

\begin{itemize}
\tightlist
\item
  Woningvoorraad
\item
  Nieuwbouw/ Sloop
\item
  Bouwkosten
\item
  Beschikbaarheid bouwlocaties
\item
  Ruimtelijkeordeningsbeleid
\end{itemize}


\end{column}

\newline

Daarna is verder ingegaan op de factoren die van invloed zijn bij de
prijsontwikkeling op de woningmarkt. Hierbij kan een onderscheidt
gemaakt worden in factoren die direct de prijsontwikkeling beïnvloeden -
en factoren die meer aanduiden of er sprake is van oververhitting op de
woningmarkt (Groot et al., 2018).

\begin{column}{0.48\textwidth}

Indicatoren prijsontwikkeling woningmarkt

\begin{itemize}
\tightlist
\item
  Gem. huizenprijzen afgelopen perioden (prijsindex)
\item
  Besteedbaar Inkomen
\item
  Particulier vermogen
\item
  Werkloosheid
\item
  Rentestanden
\item
  Consumentenvertrouwen
\item
  Fiscale behandeling, subsidies
\item
  Toegang tot krediet
\item
  Financiering hypotheek of eigen geld
\item
  Voorkeuren van kopers
\item
  Percentage huur/ koop woningen
\end{itemize}


\end{column}

\begin{column}{0.48\textwidth}

Indicatoren oververhitting woningmarkt

\begin{itemize}
\tightlist
\item
  Gem. huurprijzen (berekenen verhouding koop/huur)
\item
  Aantal woningen dat te koop staat
\item
  Gem. duur dat een woning te koop staat
\item
  Verschil tussen aanbodprijs en transactieprijs
\item
  Aanbod huizen
\end{itemize}


\end{column}

\newline

Het aantal transacties wordt verondersteld met name bepaald te worden
door het aantal huishoudens. Daarnaast kan door krapte op de woningmarkt
een schaarste ontstaan waardoor de woningmarkt `op slot' geraakt.
Tijdelijk kunnen woningen ook te duur worden wanneer huiseigenaren hun
prijzen niet verlagen in tijden van laagconjunctuur. Het
consumentenvertrouwen speelt hierbij een belangrijke rol.

Indicatoren aantal transacties

\begin{itemize}
\tightlist
\item
  Aanbod huizen
\item
  Toegang tot krediet
\item
  Werkloosheid
\item
  Consumentenvertrouwen
\item
  Verschil aanbodprijs en transactieprijs
\item
  Aantal dagen aangeboden
\end{itemize}

\hypertarget{data}{%
\section{Data}\label{data}}

De gewenste informatie is opgevraagd bij de statistici (via Reuters
Datastream) en later aangevuld met open data van het CBS (Statline).
Voor het inlezen van de (spreadsheet) Reutersdata zijn een aantal
functies ontwikkeld. Om de data van het CBS gemakkelijker in te kunnen
lezen is een routine ontwikkeld die aan de hand van metadata de
gevraagde data inleest. Routines voor het lezen en opschonen van de
data, zijn opgeslagen in separate R files.

De data die via die routines is opgehaald en opgeschoond is tussentijds
opgeslagen in Rdata formaat - en kan voor de analyse hierdoor direct
worden ingelezen. Op deze wijze is een scheiding aangebracht tussen de
data-acquisitie en de analyse van de resultaten.

Ter beperking van de omvang van dit onderzoek is een selecties gemaakt
op basis van de beschikbare data.

Gegevens over de verhouding tussen huur en koopprijzen, het aantal te
koop staande huizen, de gem. tijd dat een woning te koop staat en de
prijsverschillen tussen aanbodprijs en transactieprijs zeggen iets over
de mate van oververhitting van de markt (zie Groot et al. (2018)). Ik
beschik echter niet over voldoende data om hier een tijdsanalyse over
uit te voeren. Deze data is verder buiten deze analyse gelaten.

Voor de indicatoren over het aantal transacties is in eerste instantie
gekeken naar tijdreeksanalyse van de transacties zelf (autocorrelatie).
Daarna is tevens gekeken naar de mate waarin we de transactievolumes
kunnen voorstellen aan de hand van economische indicatoren.

\begin{Shaded}
\begin{Highlighting}[]
\KeywordTok{library}\NormalTok{ (tidyverse)          }\CommentTok{# Tidyverse utilities (Wickham et al. 2019)}
\KeywordTok{library}\NormalTok{ (here)               }\CommentTok{# here: A Simpler Way to Find Your Files (Müller 2020))}
\KeywordTok{library}\NormalTok{ (lubridate)          }\CommentTok{# Dates and Times Made Easy (Grolemund, Wickham 2011))}
\KeywordTok{library}\NormalTok{ (reshape2)           }\CommentTok{# Reshaping Data (Hadley Wickham 2007)}
\KeywordTok{library}\NormalTok{ (knitr)              }\CommentTok{# A General-Purpose Package for Dynamic Report Generation (Xie 2021)}
\KeywordTok{library}\NormalTok{ (kableExtra)         }\CommentTok{# Construct Complex Table with 'kable' and Pipe Syntax(Zhu 2021)}
\KeywordTok{library}\NormalTok{ (viridis)            }\CommentTok{# Default Color Maps from 'matplotlib' (Garnier 2018)}
\KeywordTok{library}\NormalTok{ (forcats)            }\CommentTok{# Tools for Working with Categorical Variables (Factors) (Wickham 2021)}
\KeywordTok{library}\NormalTok{ (forecast)           }\CommentTok{# Forecasting functions for time series and linear models (Hyndman et al. 2021)}
\KeywordTok{library}\NormalTok{ (fpp2)               }\CommentTok{# Data for "Forecasting: Principles and Practice" (Hyndman 2020)}
\KeywordTok{library}\NormalTok{ (zoo)                }\CommentTok{# S3 Infrastructure for Regular and Irregular Time}
                             \CommentTok{# Series (Zeileis and Grothendieck 2005)}
\KeywordTok{library}\NormalTok{ (urca)               }\CommentTok{# Analysis of Integrated and Cointegrated }
                             \CommentTok{# Time Series with R (Pfaff 2008)}
\KeywordTok{library}\NormalTok{ (cowplot)            }\CommentTok{# Streamlined Plot Theme and Plot Annotations }
                             \CommentTok{# for 'ggplot2' (Claus O. Wilke 2020)}
\KeywordTok{library}\NormalTok{ (sjPlot)             }\CommentTok{# Data Visualization for Statistics in Social Science (Lüdecke 2021)}
\KeywordTok{library}\NormalTok{ (sf)                 }\CommentTok{# Standardized Support for Spatial Vector Data (Pebesma 2018)}
\KeywordTok{library}\NormalTok{ (rsample)            }\CommentTok{# General Resampling Infrastructure (Silge, Chow, Kuhn and Wickham 2021)}
\KeywordTok{library}\NormalTok{ (earth)              }\CommentTok{# Multivariate Adaptive Regression Splines (Milborrow, 2020)}
\KeywordTok{library}\NormalTok{ (caret)              }\CommentTok{# Classification and Regression Training (Kuhn 2020)}
\KeywordTok{library}\NormalTok{ (vip)                }\CommentTok{# Variable Importance Plots (Greenwell and Boehmke 2020)}
\KeywordTok{library}\NormalTok{ (pdp)                }\CommentTok{# Constructing Partial Dependence Plots (Greenwell 2017)}
\KeywordTok{library}\NormalTok{ (timetk)             }\CommentTok{# A Tool Kit for Working with Time Series}
                             \CommentTok{# in R (Dancho and Vaughan 2021)}
\KeywordTok{library}\NormalTok{ (corrplot)           }\CommentTok{# Visualization of a Correlation Matrix (Wei and Simko 2017)}
\KeywordTok{library}\NormalTok{ (stargazer)          }\CommentTok{# Well-formatted regression summary statistics (Hlavac 2018) }
\KeywordTok{library}\NormalTok{ (dynlm)              }\CommentTok{# Dynamic linair models and time series regression (Zeileis 2019)}
\KeywordTok{library}\NormalTok{ (broom)              }\CommentTok{# Convert Statistical Objects into Tidy Tibbles (Robinson, Hayes and Couch 2021)}
\KeywordTok{library}\NormalTok{ (pacman)             }\CommentTok{# Package Management for R (Rinker, Kurkiewicz 2017)}


\CommentTok{# Export Citations as Bibtex (.bib)}
\NormalTok{knitr}\OperatorTok{::}\KeywordTok{write_bib}\NormalTok{(}\DataTypeTok{file=}\StringTok{"Bibliography of packages.bib"}\NormalTok{)}
\CommentTok{# Table (.csv) with all information on the packages}
\NormalTok{appendix_packages <-}\StringTok{ }\KeywordTok{data.frame}\NormalTok{(}\DataTypeTok{Packagename =} \KeywordTok{character}\NormalTok{(),}
                                \DataTypeTok{Version =} \KeywordTok{character}\NormalTok{(),}
                                \DataTypeTok{Maintainer =} \KeywordTok{character}\NormalTok{())}


\ControlFlowTok{for}\NormalTok{ (pkg }\ControlFlowTok{in} \KeywordTok{p_loaded}\NormalTok{())\{}
\NormalTok{  appendix_packages <-}\StringTok{ }\NormalTok{appendix_packages }\OperatorTok\StringTok{ }
\StringTok{    }\KeywordTok{add_row}\NormalTok{(}
    \DataTypeTok{Packagename =}\NormalTok{ pkg,}
    \DataTypeTok{Version =} \KeywordTok{as.character}\NormalTok{(}\KeywordTok{packageVersion}\NormalTok{(pkg)),}
    \DataTypeTok{Maintainer =} \KeywordTok{maintainer}\NormalTok{(pkg)}
\NormalTok{  )}
\NormalTok{\}}

\CommentTok{# Write citations for R packages }
\KeywordTok{write.csv}\NormalTok{(}\DataTypeTok{x =}\NormalTok{ appendix_packages, }\DataTypeTok{file =} \StringTok{"List_of_packages.csv"}\NormalTok{, }\DataTypeTok{row.names =}\NormalTok{ F)}
\end{Highlighting}
\end{Shaded}

\begin{Shaded}
\begin{Highlighting}[]
\CommentTok{# Loading the datasets}
\CommentTok{# Data is read in module readdata.R}
\CommentTok{# Data is then stored in Rdata format - and is loaded here for further analyses.}

\NormalTok{dfs<-}\KeywordTok{list.files}\NormalTok{(}\KeywordTok{here}\NormalTok{(}\StringTok{"data"}\NormalTok{,}\StringTok{"tidy"}\NormalTok{), }\DataTypeTok{pattern =} \StringTok{"*.Rdata"}\NormalTok{)}
\ControlFlowTok{for}\NormalTok{(d }\ControlFlowTok{in}\NormalTok{ dfs) \{}
  \KeywordTok{load}\NormalTok{(}\DataTypeTok{file=} \KeywordTok{here}\NormalTok{(}\StringTok{"data"}\NormalTok{,}\StringTok{"tidy"}\NormalTok{,d ))}
\NormalTok{\}}

\CommentTok{# Set default theme for all plots}
\KeywordTok{theme_set}\NormalTok{(}\KeywordTok{theme_minimal}\NormalTok{())}
\end{Highlighting}
\end{Shaded}

\hypertarget{analyse}{%
\section{Analyse}\label{analyse}}

\hypertarget{prijsontwikkeling-woningmarkt}{%
\subsection{Prijsontwikkeling
woningmarkt}\label{prijsontwikkeling-woningmarkt}}

Prijzen van koopwoningen zijn de laatste jaren erg sterk toegenomen. We
zien kleine verschillen per type woning. Waarbij met name appartementen
de laatste tijd relatief duurder zijn geworden.

\begin{Shaded}
\begin{Highlighting}[]
\NormalTok{prijsindex_type_woning }\OperatorTok
\StringTok{  }\KeywordTok{filter}\NormalTok{(type_woning }\OperatorTok{!=}\StringTok{ "Totaal woningen"}\NormalTok{) }\OperatorTok
\StringTok{  }\KeywordTok{filter}\NormalTok{(type_woning }\OperatorTok{!=}\StringTok{ "EGW"}\NormalTok{) }\OperatorTok
\StringTok{  }\KeywordTok{ggplot}\NormalTok{(}\KeywordTok{aes}\NormalTok{(}\DataTypeTok{x=}\NormalTok{date, }\DataTypeTok{y=}\NormalTok{prijsindex_type_woning)) }\OperatorTok{+}
\StringTok{  }\KeywordTok{geom_line}\NormalTok{(}\KeywordTok{aes}\NormalTok{(}\DataTypeTok{color =}\NormalTok{ type_woning)) }\OperatorTok{+}
\StringTok{  }\KeywordTok{facet_wrap}\NormalTok{(}\OperatorTok{~}\StringTok{ }\NormalTok{type_woning) }\OperatorTok{+}
\StringTok{  }\KeywordTok{labs}\NormalTok{(}
    \DataTypeTok{title =} \StringTok{"Prijsindex Bestaande Koopwoningen (PBK)"}\NormalTok{,}
    \DataTypeTok{y =} \StringTok{""}\NormalTok{,}
    \DataTypeTok{x =} \StringTok{''}\NormalTok{,}
    \DataTypeTok{color =} \StringTok{'Type woning'}
\NormalTok{    ) }\OperatorTok{+}
\StringTok{  }\KeywordTok{theme}\NormalTok{(}\DataTypeTok{legend.position =} \StringTok{"none"}\NormalTok{)}
\end{Highlighting}
\end{Shaded}

\includegraphics{data_exploration_files/figure-latex/prijsindex-1.pdf}

\hypertarget{vraag-naar-woningen}{%
\subsubsection{Vraag naar woningen}\label{vraag-naar-woningen}}

Bij de factoren die de vraag naar woningen bepalen kijken we naar de
ontwikkeling van het aantal huishoudens en de samenstelling van de
huishoudens. Nederlanders worden steeds ouder, en gezinnen worden
kleiner. Deze ontwikkeling leidt mogelijk tot een verschuiving in de
vraag - van eengezinswoningen naar kleinere woningen.

\begin{Shaded}
\begin{Highlighting}[]
\NormalTok{bevolking }\OperatorTok\StringTok{ }
\StringTok{  }\KeywordTok{select}\NormalTok{ (date, kl_}\DecValTok{20}\NormalTok{_jaar, v20_}\DecValTok{40}\NormalTok{_jaar, v40_}\DecValTok{65}\NormalTok{_jaar, v65_}\DecValTok{80}\NormalTok{_jaar, gd_}\DecValTok{80}\NormalTok{_jaar) }\OperatorTok\StringTok{ }
\StringTok{  }\KeywordTok{filter}\NormalTok{ (date }\OperatorTok{>=}\StringTok{ }\KeywordTok{as.Date}\NormalTok{(}\StringTok{"1995-01-01"}\NormalTok{)) }\OperatorTok\StringTok{ }
\StringTok{  }\KeywordTok{melt}\NormalTok{ (}\DataTypeTok{id =} \KeywordTok{c}\NormalTok{(}\StringTok{'date'}\NormalTok{)) }\OperatorTok\StringTok{ }
\StringTok{  }\KeywordTok{mutate}\NormalTok{ ( }\DataTypeTok{variable =} \KeywordTok{factor}\NormalTok{(variable), }
           \DataTypeTok{variable =} \KeywordTok{factor}\NormalTok{(variable, }\DataTypeTok{levels =} \KeywordTok{rev}\NormalTok{(}\KeywordTok{levels}\NormalTok{(variable)))) }\OperatorTok\StringTok{ }
\StringTok{  }\KeywordTok{mutate}\NormalTok{ (}\DataTypeTok{value =}\NormalTok{ value }\OperatorTok{/}\StringTok{ }\DecValTok{1000}\NormalTok{) }\OperatorTok\StringTok{ }
\StringTok{  }\KeywordTok{ggplot}\NormalTok{ ( }\KeywordTok{aes}\NormalTok{( }\DataTypeTok{x =}\NormalTok{ date, }\DataTypeTok{y =}\NormalTok{ value, }\DataTypeTok{fill =}\NormalTok{ variable) ) }\OperatorTok{+}
\StringTok{  }\KeywordTok{geom_area}\NormalTok{(}\DataTypeTok{color =} \StringTok{"darkgrey"}\NormalTok{, }\DataTypeTok{stat =} \StringTok{"identity"}\NormalTok{, }\DataTypeTok{alpha =} \FloatTok{0.8}\NormalTok{) }\OperatorTok{+}
\StringTok{  }\KeywordTok{labs}\NormalTok{(}
    \DataTypeTok{title =} \StringTok{"Bevolkingsgroei"}\NormalTok{,}
    \DataTypeTok{subtitle =} \StringTok{"1995 tot 2020"}\NormalTok{,}
    \DataTypeTok{y =} \StringTok{"Aantal (x 1000)"}\NormalTok{,}
    \DataTypeTok{x =} \StringTok{''}\NormalTok{,}
    \DataTypeTok{caption =} \StringTok{"Bron: CBS, Bevolking; kerncijfers"}
\NormalTok{    )  }
\end{Highlighting}
\end{Shaded}

\includegraphics{data_exploration_files/figure-latex/population_growth-1.pdf}

De groei van de Nederlandse bevolking wordt in de laatste jaren steeds
meer bepaald door migratie. Het geboorteoverschot (aantal nieuw
geborenen -/- aantal sterfgevallen is nog nooit zo laag geweest).
Mogelijk leidt dit tot een verschuiving in de vraag naar meer woningen
in de steden. Dit zou verder onderzocht moeten worden.

\begin{Shaded}
\begin{Highlighting}[]
\NormalTok{bevolking }\OperatorTok
\StringTok{  }\KeywordTok{select}\NormalTok{ (date, geboorteoverschot, migratiesaldo) }\OperatorTok\StringTok{ }
\StringTok{  }\KeywordTok{filter}\NormalTok{ (date }\OperatorTok{>=}\StringTok{ }\KeywordTok{as.Date}\NormalTok{(}\StringTok{"1970-01-01"}\NormalTok{)) }\OperatorTok\StringTok{ }
\StringTok{  }\KeywordTok{melt}\NormalTok{ (}\DataTypeTok{id =} \KeywordTok{c}\NormalTok{(}\StringTok{'date'}\NormalTok{)) }\OperatorTok\StringTok{ }
\StringTok{  }\KeywordTok{mutate}\NormalTok{ (}\DataTypeTok{value =}\NormalTok{ value }\OperatorTok{/}\StringTok{ }\DecValTok{1000}\NormalTok{) }\OperatorTok\StringTok{ }
\StringTok{  }\NormalTok{drop_na }\OperatorTok\StringTok{ }
\StringTok{  }\KeywordTok{ggplot}\NormalTok{(}\KeywordTok{aes}\NormalTok{ (}\DataTypeTok{x =}\NormalTok{ date, }\DataTypeTok{y=}\NormalTok{value, }\DataTypeTok{fill =}\NormalTok{ variable)) }\OperatorTok{+}
\StringTok{  }\KeywordTok{geom_bar}\NormalTok{(}\DataTypeTok{stat=}\StringTok{"identity"}\NormalTok{) }\OperatorTok{+}
\StringTok{  }\KeywordTok{labs}\NormalTok{(}
    \DataTypeTok{title =} \StringTok{"Bevolkingsgroei"}\NormalTok{,}
    \DataTypeTok{y =} \StringTok{"Aantal (x 1000)"}\NormalTok{,}
    \DataTypeTok{x =} \StringTok{''}\NormalTok{,}
    \DataTypeTok{fill =} \StringTok{''}\NormalTok{,}
    \DataTypeTok{caption =} \StringTok{"Bron: CBS, Bevolking; kerncijfers"}
\NormalTok{    ) }
\end{Highlighting}
\end{Shaded}

\includegraphics{data_exploration_files/figure-latex/population_details-1.pdf}

Het CBS maakt regelmatig een prognose van de verwachte groei. Hier zien
we dat deze trend zich de aankomende jaren voortzet. De groei komt
voornamelijk door de groei van het aantal eenpersoonshuishoudens. De
laatste prognoses zijn overigens van 2019. In de prognose zien we een
afvlakking van de groei van het aantal huishoudens rond 2040.

\begin{Shaded}
\begin{Highlighting}[]
\NormalTok{aantal_hh <-}\StringTok{ }
\StringTok{  }\NormalTok{bevolking }\OperatorTok\StringTok{ }
\StringTok{  }\KeywordTok{filter}\NormalTok{( date }\OperatorTok{>=}\StringTok{ }\KeywordTok{as.Date}\NormalTok{(}\StringTok{"1970-01-01"}\NormalTok{)) }\OperatorTok
\StringTok{  }\KeywordTok{select}\NormalTok{ (date,}
\NormalTok{            eenpersoonshuishoudens,}
\NormalTok{            meerpersoonshuishoudens) }\OperatorTok\StringTok{ }
\StringTok{  }\KeywordTok{melt}\NormalTok{(}\DataTypeTok{id =} \KeywordTok{c}\NormalTok{(}\StringTok{'date'}\NormalTok{)) }\OperatorTok\StringTok{ }
\StringTok{  }\KeywordTok{set_names}\NormalTok{ (}\StringTok{'date'}\NormalTok{,}\StringTok{'type_huishouden'}\NormalTok{, }\StringTok{'aantal'}\NormalTok{)}
  
  
\NormalTok{prognose_hh <-}
\StringTok{  }\NormalTok{prognose_huishoudens_}\DecValTok{2019} \OperatorTok\StringTok{ }
\StringTok{  }\KeywordTok{select}\NormalTok{ (date,}
\NormalTok{            eenpersoonshuishoudens,}
\NormalTok{            meerpersoonshuishoudens) }\OperatorTok\StringTok{ }
\StringTok{  }\KeywordTok{melt}\NormalTok{(}\DataTypeTok{id =} \KeywordTok{c}\NormalTok{(}\StringTok{'date'}\NormalTok{)) }\OperatorTok\StringTok{ }
\StringTok{  }\KeywordTok{set_names}\NormalTok{ (}\StringTok{'date'}\NormalTok{,}\StringTok{'type_huishouden'}\NormalTok{, }\StringTok{'prognose'}\NormalTok{) }\OperatorTok\StringTok{ }
\StringTok{  }\KeywordTok{mutate}\NormalTok{(}\DataTypeTok{prognose =}\NormalTok{ prognose }\OperatorTok{/}\StringTok{ }\DecValTok{1000}\NormalTok{)}

\NormalTok{aantal_hh }\OperatorTok\StringTok{ }
\StringTok{  }\KeywordTok{ggplot}\NormalTok{(}\KeywordTok{aes}\NormalTok{ (}\DataTypeTok{x =}\NormalTok{ date, }\DataTypeTok{y=}\NormalTok{value, }\DataTypeTok{fill =}\NormalTok{ variable)) }\OperatorTok{+}
\StringTok{  }\KeywordTok{geom_area}\NormalTok{(}\KeywordTok{aes}\NormalTok{(}\DataTypeTok{y=}\NormalTok{aantal,  }
                 \DataTypeTok{fill=}\NormalTok{ type_huishouden}
\NormalTok{                 )) }\OperatorTok{+}
\StringTok{  }\KeywordTok{geom_area}\NormalTok{(}\DataTypeTok{data =}\NormalTok{ prognose_hh,}
            \KeywordTok{aes}\NormalTok{(}\DataTypeTok{y=}\NormalTok{prognose,  }
                \DataTypeTok{fill=}\NormalTok{ type_huishouden,}
                \DataTypeTok{alpha =} \FloatTok{0.1}
\NormalTok{                 )) }\OperatorTok{+}
\StringTok{  }\KeywordTok{geom_vline}\NormalTok{(}\DataTypeTok{xintercept=}\KeywordTok{max}\NormalTok{(aantal_hh}\OperatorTok{$}\NormalTok{date), }\DataTypeTok{color=}\StringTok{"darkgray"}\NormalTok{, }\DataTypeTok{size=}\DecValTok{1}\NormalTok{) }\OperatorTok{+}
\StringTok{  }\KeywordTok{labs}\NormalTok{(}
    \DataTypeTok{title =} \StringTok{"Prognose aantal huishoudens"}\NormalTok{,}
    \DataTypeTok{y =} \StringTok{"Aantal (x 1000)"}\NormalTok{,}
    \DataTypeTok{x =} \StringTok{''}\NormalTok{,}
    \DataTypeTok{caption =} \StringTok{"Bron: CBS, Bevolking; kerncijfers"}
\NormalTok{  ) }\OperatorTok{+}
\StringTok{  }\KeywordTok{theme}\NormalTok{(}\DataTypeTok{legend.title =} \KeywordTok{element_blank}\NormalTok{()) }\OperatorTok{+}
\StringTok{  }\KeywordTok{guides}\NormalTok{(}\DataTypeTok{alpha =} \OtherTok{FALSE}\NormalTok{) }\OperatorTok{+}
\StringTok{  }\KeywordTok{annotate}\NormalTok{(}\StringTok{"text"}\NormalTok{, }\DataTypeTok{x =} \KeywordTok{max}\NormalTok{(aantal_hh}\OperatorTok{$}\NormalTok{date), }\DataTypeTok{y =} \DecValTok{0}\NormalTok{,}
           \DataTypeTok{label =} \KeywordTok{c}\NormalTok{(}\StringTok{" Prognose"}\NormalTok{) , }\DataTypeTok{color=}\StringTok{"black"}\NormalTok{, }\DataTypeTok{hjust =} \DecValTok{0}\NormalTok{, }\DataTypeTok{vjust =} \DecValTok{1}\NormalTok{,}
           \DataTypeTok{size=}\DecValTok{3}\NormalTok{)}
\end{Highlighting}
\end{Shaded}

\includegraphics{data_exploration_files/figure-latex/population_prognose-1.pdf}

\hypertarget{aanbod-van-woningen}{%
\subsubsection{Aanbod van woningen}\label{aanbod-van-woningen}}

Wanneer we een wat langere termijn trend kijken naar het aantal
gerealiseerde nieuwbouwhuizen zien we dat het aantal nieuwbouwhuizen
eigenlijk nu niet bijzonder hoog is. Ondanks dat er meer dan 70.000
huizen bijgebouwd wordt per jaar is dat vergeleken met de totale
woningvoorraad eigenlijk niet bijzonder veel. Wel opvallend is dat de
laatste jaren er meer overige toevoegingen en overige onttrekkingen
geregistreerd worden. Mogelijk geeft dit aan dat de ruimte voor
nieuwbouw schaarser wordt waardoor creatiever wordt omgegaan met de
beschikbare ruimte.

\begin{Shaded}
\begin{Highlighting}[]
\NormalTok{voorraad_woningen }\OperatorTok
\StringTok{  }\KeywordTok{select}\NormalTok{(date,}
\NormalTok{         nieuwbouw,}
\NormalTok{         overige_toevoegingen,}
\NormalTok{         sloop,}
\NormalTok{         Overige_onttrekking,}
\NormalTok{         correctie}
\NormalTok{         ) }\OperatorTok
\StringTok{  }\KeywordTok{mutate}\NormalTok{ (}\DataTypeTok{sloop =} \OperatorTok{-}\StringTok{ }\NormalTok{sloop ) }\OperatorTok\StringTok{ }
\StringTok{  }\KeywordTok{mutate}\NormalTok{ (}\DataTypeTok{Overige_onttrekking =} \OperatorTok{-}\StringTok{ }\NormalTok{Overige_onttrekking) }\OperatorTok\StringTok{ }
\StringTok{  }\KeywordTok{mutate}\NormalTok{ (}\DataTypeTok{correctie =} \OperatorTok{-}\StringTok{ }\NormalTok{correctie) }\OperatorTok\StringTok{ }
\StringTok{  }\KeywordTok{filter}\NormalTok{( date }\OperatorTok{>=}\StringTok{ }\KeywordTok{as.Date}\NormalTok{(}\StringTok{"1970-01-01"}\NormalTok{)) }\OperatorTok\StringTok{ }
\StringTok{  }\KeywordTok{melt}\NormalTok{ (}\DataTypeTok{id =} \KeywordTok{c}\NormalTok{(}\StringTok{'date'}\NormalTok{)) }\OperatorTok\StringTok{ }
\StringTok{  }\KeywordTok{mutate}\NormalTok{ (}\DataTypeTok{value =}\NormalTok{ value }\OperatorTok{/}\StringTok{ }\DecValTok{1000}\NormalTok{) }\OperatorTok\StringTok{ }
\StringTok{  }\KeywordTok{ggplot}\NormalTok{(}\KeywordTok{aes}\NormalTok{(}\DataTypeTok{x=}\NormalTok{date, }\DataTypeTok{y =}\NormalTok{ value, }\DataTypeTok{fill =}\NormalTok{ variable )) }\OperatorTok{+}
\StringTok{  }\KeywordTok{geom_bar}\NormalTok{(}\DataTypeTok{stat=}\StringTok{"identity"}\NormalTok{) }\OperatorTok{+}
\StringTok{  }\KeywordTok{geom_hline}\NormalTok{(}\DataTypeTok{yintercept  =} \DecValTok{0}\NormalTok{, }\DataTypeTok{color =} \StringTok{"steelblue"}\NormalTok{) }\OperatorTok{+}
\StringTok{  }\KeywordTok{labs}\NormalTok{ ( }\DataTypeTok{title =} \StringTok{"Ontwikkelingen woningvoorraad"}\NormalTok{,}
         \DataTypeTok{y =} \StringTok{"Aantal (x 1000)"}\NormalTok{,}
         \DataTypeTok{x =} \StringTok{"Jaar"}\NormalTok{) }\OperatorTok{+}\StringTok{ }
\StringTok{    }\KeywordTok{guides}\NormalTok{(}\DataTypeTok{fill =} \KeywordTok{guide_legend}\NormalTok{(}\DataTypeTok{title =} \StringTok{''}\NormalTok{))}
\end{Highlighting}
\end{Shaded}

\includegraphics{data_exploration_files/figure-latex/houses-1.pdf}

\hypertarget{samenhang-vraag-en-aanbod}{%
\subsubsection{Samenhang Vraag en
aanbod}\label{samenhang-vraag-en-aanbod}}

\begin{Shaded}
\begin{Highlighting}[]
\CommentTok{#Aantal huishoudens: NLHSTOTP Thomson Reuters}
\NormalTok{df <-}\StringTok{ }
\StringTok{  }\NormalTok{aantal_huishoudens_per_jaar }\OperatorTok
\StringTok{  }\KeywordTok{full_join}\NormalTok{(voorraad_woningen, }\DataTypeTok{by =} \KeywordTok{c}\NormalTok{(}\StringTok{"date"}\NormalTok{ =}\StringTok{ "date"}\NormalTok{)) }\OperatorTok
\StringTok{  }\KeywordTok{select}\NormalTok{ (date, }
\NormalTok{          aantal_huishoudens, }
\NormalTok{          voorraad) }\OperatorTok
\StringTok{  }\KeywordTok{set_names}\NormalTok{(}\StringTok{'date'}\NormalTok{,}
            \StringTok{'huishoudens'}\NormalTok{,}
            \StringTok{'woningvoorraad'}\NormalTok{ ) }\OperatorTok
\StringTok{  }\KeywordTok{mutate}\NormalTok{(}\DataTypeTok{te_kort =}\NormalTok{ (woningvoorraad }\OperatorTok{-}\StringTok{ }\NormalTok{huishoudens)) }\OperatorTok
\StringTok{  }\KeywordTok{filter}\NormalTok{(date }\OperatorTok{>}\StringTok{ }\KeywordTok{as.Date}\NormalTok{(}\StringTok{"2000-01-01"}\NormalTok{)) }\OperatorTok
\StringTok{  }\KeywordTok{melt}\NormalTok{(}\DataTypeTok{id =} \KeywordTok{c}\NormalTok{(}\StringTok{'date'}\NormalTok{)) }

\CommentTok{# Line plot}
\NormalTok{lp<-}\StringTok{ }
\StringTok{  }\NormalTok{df }\OperatorTok
\StringTok{  }\KeywordTok{filter}\NormalTok{ (variable }\OperatorTok{!=}\StringTok{ 'te_kort'}\NormalTok{) }\OperatorTok
\StringTok{  }\KeywordTok{ggplot}\NormalTok{(}\KeywordTok{aes}\NormalTok{ (}\DataTypeTok{x =}\NormalTok{ date)) }\OperatorTok{+}
\StringTok{  }\KeywordTok{geom_line}\NormalTok{(}\KeywordTok{aes}\NormalTok{(}\DataTypeTok{y=}\NormalTok{value, }\DataTypeTok{color =}\NormalTok{ variable)) }\OperatorTok{+}
\StringTok{  }\KeywordTok{labs}\NormalTok{(}
    \DataTypeTok{title =} \StringTok{"Aantal huishoudens vs voorraad woningen"}\NormalTok{,}
    \DataTypeTok{y =} \StringTok{"Aantal (x 1000)"}\NormalTok{,}
    \DataTypeTok{x =} \StringTok{''}\NormalTok{,}
    \DataTypeTok{color =} \StringTok{''}
\NormalTok{    ) }

\CommentTok{# Bar plot}
\NormalTok{bp <-}\StringTok{ }
\StringTok{  }\NormalTok{df }\OperatorTok
\StringTok{  }\KeywordTok{filter}\NormalTok{ (variable }\OperatorTok{==}\StringTok{ 'te_kort'}\NormalTok{) }\OperatorTok
\StringTok{  }\KeywordTok{ggplot}\NormalTok{(}\KeywordTok{aes}\NormalTok{ (}\DataTypeTok{x =}\NormalTok{ date)) }\OperatorTok{+}
\StringTok{  }\KeywordTok{geom_bar}\NormalTok{(}\DataTypeTok{stat =} \StringTok{"identity"}\NormalTok{ , }
           \KeywordTok{aes}\NormalTok{(}\DataTypeTok{y =}\NormalTok{ value, }\DataTypeTok{fill =} \KeywordTok{factor}\NormalTok{(variable, }\DataTypeTok{labels =} \KeywordTok{c}\NormalTok{(}\StringTok{"Te Kort woningen"}\NormalTok{)))) }\OperatorTok{+}
\StringTok{  }\KeywordTok{labs}\NormalTok{ ( }\DataTypeTok{y =} \StringTok{"Te kort (x 1000)"}\NormalTok{,}
         \DataTypeTok{x =} \StringTok{""}\NormalTok{,}
         \DataTypeTok{fill =} \StringTok{""}\NormalTok{) }\OperatorTok{+}
\StringTok{  }\KeywordTok{theme}\NormalTok{( }\DataTypeTok{axis.text.x =} \KeywordTok{element_blank}\NormalTok{(),}
         \DataTypeTok{axis.ticks =} \KeywordTok{element_blank}\NormalTok{()) }
  
\CommentTok{# Plot line and bar plot together}
\NormalTok{cowplot}\OperatorTok{::}\KeywordTok{plot_grid}\NormalTok{(lp, bp, }\DataTypeTok{ncol=}\DecValTok{1}\NormalTok{, }\DataTypeTok{rel_heights =} \KeywordTok{c}\NormalTok{(}\DecValTok{2}\NormalTok{, }\DecValTok{1}\NormalTok{), }\DataTypeTok{align =} \StringTok{'v'}\NormalTok{)}
\end{Highlighting}
\end{Shaded}

\includegraphics{data_exploration_files/figure-latex/shortage_houses-1.pdf}

In bovenstaande grafiek zien we dat het aantal woningen al decennialang
gelijke tred houdt met het aantal woningen. Er is daarbij telkens wel
een tekort zichtbaar. Dit is veel kleiner dan het door het CBS
gerapporteerde tekort van 331.000 woningen {[}CBS2020b{]}. In deze
studie is er geen verklaring gevonden voor dit verschil. In de verdere
analyse wordt uitgegaan van de tekorten zoals hier berekend. Dit is het
verschil tussen het aantal huishoudens en de voorraad woningen.

\hypertarget{indicatoren-huizenprijzen}{%
\subsubsection{Indicatoren
Huizenprijzen}\label{indicatoren-huizenprijzen}}

\begin{itemize}
\tightlist
\item
  Gem. huizenprijzen (prijsindex)
\item
  Besteedbaar Inkomen
\item
  Particulier vermogen
\item
  Werkloosheid
\item
  Rentestanden
\item
  Consumentenvertrouwen
\item
  Financiering hypotheek of eigen geld
\item
  Fiscale behandeling, subsidies
\item
  Toegang tot krediet
\item
  Voorkeuren van kopers
\item
  Percentage huur / koop woningen
\end{itemize}

Vanwege de (structurele) te korten op de woningmarkt lijken
huizenprijzen in belangrijke mate beïnvloed te worden door de
financieringsmogelijkheden van de kopers. De rente is historisch laag,
de inkomens zijn gestegen en het eigen vermogen is gegroeid. Dit geeft
ruimte voor hogere marktprijzen. De fiscale behandeling en regelgeving
speelt hier een belangrijke rol, zowel ruimte scheppend - verruimen van
de leenmogelijkheden van tweeverdieners, als beperkend - beperking
aftrekbaarheid tot 100\% financiering.

\begin{Shaded}
\begin{Highlighting}[]
\CommentTok{# House price index existing homes}
\NormalTok{prijsindex_woningen_lt }\OperatorTok\StringTok{ }
\KeywordTok{ggplot}\NormalTok{(}\KeywordTok{aes}\NormalTok{(}\DataTypeTok{x =}\NormalTok{ date)) }\OperatorTok{+}
\KeywordTok{geom_line}\NormalTok{(}\KeywordTok{aes}\NormalTok{( }\DataTypeTok{y =}\NormalTok{ huisprijsindex), }\DataTypeTok{color =} \StringTok{'steelblue'}\NormalTok{) }\OperatorTok{+}
\KeywordTok{labs}\NormalTok{(}
    \DataTypeTok{title =} \StringTok{"Huizenprijzen"}\NormalTok{,}
    \DataTypeTok{subtitle =} \StringTok{"Index (2015 = 100)"}\NormalTok{,}
    \DataTypeTok{y =} \StringTok{""}\NormalTok{,}
    \DataTypeTok{x =} \StringTok{""}
\NormalTok{) }
\end{Highlighting}
\end{Shaded}

\includegraphics{data_exploration_files/figure-latex/leading_indicators-1.pdf}

\begin{Shaded}
\begin{Highlighting}[]
\CommentTok{# GDP}
\NormalTok{gdp_lt }\OperatorTok\StringTok{ }
\StringTok{  }\KeywordTok{ggplot}\NormalTok{( }\KeywordTok{aes}\NormalTok{(}\DataTypeTok{x=}\NormalTok{date, }\DataTypeTok{y=}\NormalTok{gdp)) }\OperatorTok{+}
\StringTok{    }\KeywordTok{geom_line}\NormalTok{(}\DataTypeTok{color =} \StringTok{'steelblue'}\NormalTok{) }\OperatorTok{+}
\StringTok{    }\KeywordTok{labs}\NormalTok{(}
      \DataTypeTok{title =} \StringTok{"Ontwikkeling BBP (Bruto Binnenlands Product) (j-op-j wijzigingen)"}\NormalTok{,}
      \DataTypeTok{y =} \StringTok{""}\NormalTok{,}
      \DataTypeTok{x =} \StringTok{''}\NormalTok{,}
      \DataTypeTok{caption =} \StringTok{"Bron: OECD Economic Outlook"}
\NormalTok{      ) }
\end{Highlighting}
\end{Shaded}

\includegraphics{data_exploration_files/figure-latex/leading_indicators-2.pdf}

\begin{Shaded}
\begin{Highlighting}[]
\CommentTok{# Interest rates}
\NormalTok{hypotheek_rente }\OperatorTok
\StringTok{  }\KeywordTok{ggplot}\NormalTok{( }\KeywordTok{aes}\NormalTok{(}\DataTypeTok{x=}\NormalTok{period, }\DataTypeTok{y=}\NormalTok{hypotheek_rente)) }\OperatorTok{+}
\StringTok{  }\KeywordTok{geom_line}\NormalTok{(}\DataTypeTok{color =} \StringTok{'steelblue'}\NormalTok{)  }\OperatorTok{+}
\StringTok{  }\KeywordTok{labs}\NormalTok{(}
    \DataTypeTok{title =} \StringTok{"Hypotheekrente"}\NormalTok{,}
    \DataTypeTok{y =} \StringTok{"gemiddelde rente"}\NormalTok{,}
    \DataTypeTok{x =} \StringTok{''}
\NormalTok{    ) }
\end{Highlighting}
\end{Shaded}

\includegraphics{data_exploration_files/figure-latex/leading_indicators-3.pdf}

\begin{Shaded}
\begin{Highlighting}[]
\CommentTok{# Employment}
\NormalTok{werkgelegenheid_lt }\OperatorTok\StringTok{ }
\StringTok{  }\KeywordTok{ggplot}\NormalTok{(}\KeywordTok{aes}\NormalTok{(}\DataTypeTok{x =}\NormalTok{ period, }\DataTypeTok{y =}\NormalTok{ werkgelegenheid)) }\OperatorTok{+}
\StringTok{  }\KeywordTok{geom_line}\NormalTok{(}\DataTypeTok{color =} \StringTok{'steelblue'}\NormalTok{) }\OperatorTok{+}
\StringTok{  }\KeywordTok{labs}\NormalTok{(}
    \DataTypeTok{title =} \StringTok{"Werkgelegenheid"}\NormalTok{,}
    \DataTypeTok{y =} \StringTok{"(x 1000)"}\NormalTok{,}
    \DataTypeTok{x =} \StringTok{''}\NormalTok{,}
    \DataTypeTok{caption =} \StringTok{"Bron: OECD Economic Outlook"}
\NormalTok{  ) }
\end{Highlighting}
\end{Shaded}

\includegraphics{data_exploration_files/figure-latex/leading_indicators-4.pdf}

\begin{Shaded}
\begin{Highlighting}[]
\CommentTok{# Income}
\NormalTok{inkomen_lt }\OperatorTok\StringTok{ }
\StringTok{  }\KeywordTok{ggplot}\NormalTok{(}\KeywordTok{aes}\NormalTok{(}\DataTypeTok{x =}\NormalTok{ period, }\DataTypeTok{y =}\NormalTok{ inkomen)) }\OperatorTok{+}
\StringTok{  }\KeywordTok{geom_line}\NormalTok{(}\DataTypeTok{color =} \StringTok{'steelblue'}\NormalTok{) }\OperatorTok{+}
\StringTok{  }\KeywordTok{labs}\NormalTok{(}
    \DataTypeTok{title =} \StringTok{"Inkomen per medewerker per kwartaal"}\NormalTok{,}
    \DataTypeTok{y =} \StringTok{"EUR"}\NormalTok{,}
    \DataTypeTok{x =} \StringTok{''}\NormalTok{,}
    \DataTypeTok{caption =} \StringTok{"Bron: Oxford Economics"}
\NormalTok{  ) }
\end{Highlighting}
\end{Shaded}

\includegraphics{data_exploration_files/figure-latex/leading_indicators-5.pdf}

\begin{Shaded}
\begin{Highlighting}[]
\CommentTok{# Consumer Confidence}
\NormalTok{consumenten_vertrouwen_lt_q }\OperatorTok\StringTok{ }
\StringTok{  }\KeywordTok{mutate}\NormalTok{(}\DataTypeTok{pos =} \KeywordTok{ifelse}\NormalTok{(consumer_conf }\OperatorTok{>=}\StringTok{ }\DecValTok{0}\NormalTok{,consumer_conf, }\DecValTok{0}\NormalTok{)) }\OperatorTok\StringTok{ }
\StringTok{  }\KeywordTok{mutate}\NormalTok{(}\DataTypeTok{neg =} \KeywordTok{ifelse}\NormalTok{(consumer_conf }\OperatorTok{<}\StringTok{ }\DecValTok{0}\NormalTok{,consumer_conf, }\DecValTok{0}\NormalTok{)) }\OperatorTok\StringTok{ }
\StringTok{  }\KeywordTok{ggplot}\NormalTok{(}\KeywordTok{aes}\NormalTok{(}\DataTypeTok{x =}\NormalTok{ date)) }\OperatorTok{+}
\StringTok{  }\KeywordTok{geom_area}\NormalTok{(}\KeywordTok{aes}\NormalTok{(}\DataTypeTok{y =}\NormalTok{ pos), }\DataTypeTok{fill =} \StringTok{'steelblue'}\NormalTok{)  }\OperatorTok{+}
\StringTok{  }\KeywordTok{geom_area}\NormalTok{(}\KeywordTok{aes}\NormalTok{(}\DataTypeTok{y =}\NormalTok{ neg), }\DataTypeTok{fill =} \StringTok{'indianred2'}\NormalTok{)  }\OperatorTok{+}
\StringTok{  }\KeywordTok{geom_hline}\NormalTok{ (}\DataTypeTok{yintercept =} \DecValTok{0}\NormalTok{, }\DataTypeTok{color =} \StringTok{'steelblue'}\NormalTok{) }\OperatorTok{+}
\StringTok{  }\KeywordTok{labs}\NormalTok{(}
    \DataTypeTok{title =} \StringTok{"Indicator Consumenten vertrouwen"}\NormalTok{,}
    \DataTypeTok{y =} \StringTok{""}\NormalTok{,}
    \DataTypeTok{x =} \StringTok{""}\NormalTok{,}
    \DataTypeTok{caption =} \StringTok{"Bron: OECD Economic Outlook"}
\NormalTok{  ) }
\end{Highlighting}
\end{Shaded}

\includegraphics{data_exploration_files/figure-latex/leading_indicators-6.pdf}

Over de vermogens van huishoudens en de wijzigingen in regelgeving zijn
er helaas geen langere termijn gegevens gevonden. Een veronderstelling
van dit onderzoek was dat door het gewijzigde beleid en de stijgende
huizenprijzen het vermogen van huishoudens zou stijgen. Daardoor heeft
men ook weer meer vermogen beschikbaar voor de aanschaf van woningen wat
een prijsopdrijvend effect kan hebben. Dit is door gebrek en data helaas
niet verder te verifiëren.

\hypertarget{huizenprijzen-voorspellen}{%
\subsubsection{Huizenprijzen
voorspellen}\label{huizenprijzen-voorspellen}}

In deze studie is gekeken naar een aantal modellen om een voorspelling
te kunnen doen over de ontwikkeling van de huizenprijzen. Voor alle
modellen is de data tot en met 2015 genomen als training data. Het model
geeft daarmee een voorspelling over de jaren 2016 tot 2020. Op deze
wijze kunnen we een vergelijking maken tussen de modellen. Om de
voorspellende factoren vergelijkbaar te maken is gekeken naar de
jaar-op-jaar wijzigingen in de factoren. Voor Bruto Binnenlands Product
is dit zo aangeleverd. De overige indicatoren zijn omgerekend naar
percentuele wijziging jaar-op-jaar.

\hypertarget{correlatie-van-huisprijzen}{%
\paragraph{Correlatie van
huisprijzen}\label{correlatie-van-huisprijzen}}

We vinden de volgende correlaties tussen inkomen, werkgelegenheid, BBP
en huizenprijzen.

\begin{Shaded}
\begin{Highlighting}[]
\CommentTok{# plot correlations}
\NormalTok{corMatrix <-}\StringTok{ }\KeywordTok{round}\NormalTok{(}\KeywordTok{cor}\NormalTok{(df_rm_huizenprijzen }\OperatorTok\StringTok{ }
\StringTok{                         }\KeywordTok{select}\NormalTok{(huisprijsindex_yoy,}
\NormalTok{                                werkgelegenheid_yoy,}
\NormalTok{                                hypotheek_rente_yoy,}
\NormalTok{                                gdp,}
\NormalTok{                                consumer_conf_yoy,}
\NormalTok{                                inkomen_yoy)),}\DecValTok{5}\NormalTok{)}

\NormalTok{corCols <-}\StringTok{ }\KeywordTok{c}\NormalTok{(}\StringTok{"huisprijsindex"}\NormalTok{, }
             \StringTok{"werkgelegenheid"}\NormalTok{,}
             \StringTok{"hypotheekrente"}\NormalTok{,}
             \StringTok{"BBP"}\NormalTok{,}
             \StringTok{"Cons.Vertrouwen"}\NormalTok{, }
             \StringTok{"Inkomen"}\NormalTok{)}

\KeywordTok{colnames}\NormalTok{(corMatrix) <-}\StringTok{ }\NormalTok{corCols}
\KeywordTok{rownames}\NormalTok{(corMatrix) <-}\StringTok{ }\NormalTok{corCols}

\KeywordTok{corrplot}\NormalTok{(corMatrix,}
            \DataTypeTok{title =} \StringTok{"Correlatie huizenprijzen"}\NormalTok{,}
            \DataTypeTok{type =} \StringTok{"lower"}\NormalTok{, }
            \DataTypeTok{tl.cex =} \FloatTok{0.8}\NormalTok{,}
            \DataTypeTok{tl.col =} \StringTok{"black"}\NormalTok{, }
            \DataTypeTok{tl.srt =} \DecValTok{45}\NormalTok{,}
         \DataTypeTok{mar=}\KeywordTok{c}\NormalTok{(}\DecValTok{0}\NormalTok{, }\DecValTok{0}\NormalTok{, }\DecValTok{4}\NormalTok{, }\DecValTok{0}\NormalTok{) }\CommentTok{# http://stackoverflow.com/a/14754408/54964}
\NormalTok{         ) }
\end{Highlighting}
\end{Shaded}

\includegraphics{data_exploration_files/figure-latex/correlation_houseprices-1.pdf}

\hypertarget{regressie-huizenprijzen}{%
\paragraph{Regressie Huizenprijzen}\label{regressie-huizenprijzen}}

Regressie model zou rekening moeten houden met het tijdskarakter van dit
soort reeksen. We vinden hiervan voorbeelden in de econometrie (Hanck,
Martin, Alexander, \& Schmelzer, 2020).

In deze studie is uitgegaan van een eenvoudig model om deze
tijdseffecten mee te nemen in de regressieanalyse. Voor het lineair
model met tijdsreekscorrectie worden van alle voorspellende factoren
vier kwartalen historische waarde meegenomen als mogelijke voorspelling
van de huizenprijzen.

\begin{Shaded}
\begin{Highlighting}[]
\CommentTok{#' lagdata}
\CommentTok{#' }
\CommentTok{#' Add lags to timeseries data. We assume the time serie is represented as a dataframe with column 'date'}
\CommentTok{#' This routine will then go on and add 4 lags to all variables}
\CommentTok{#' }
\CommentTok{#' @param df }
\CommentTok{#'}
\CommentTok{#' @return df}
\CommentTok{#' @export}
\CommentTok{#'}
\CommentTok{#' @examples}
\NormalTok{lagdata <-}\StringTok{ }\ControlFlowTok{function}\NormalTok{(df)\{}
  
  
  \ControlFlowTok{for}\NormalTok{ (name }\ControlFlowTok{in} \KeywordTok{colnames}\NormalTok{(df)) \{}
    \ControlFlowTok{if}\NormalTok{ (name }\OperatorTok{!=}\StringTok{ 'date'}\NormalTok{) \{}
        \ControlFlowTok{for}\NormalTok{ (i }\ControlFlowTok{in} \DecValTok{1}\OperatorTok{:}\DecValTok{4}\NormalTok{) \{}
\NormalTok{          variable_name =}\StringTok{ }\KeywordTok{paste0}\NormalTok{(name,i)}
          \CommentTok{# add lagged data to the dataframe}
\NormalTok{          df[[variable_name]]=}\StringTok{ }\KeywordTok{lag}\NormalTok{(df[[name]],i)}
\NormalTok{        \}}
\NormalTok{    \}}
\NormalTok{  \}}
\KeywordTok{return}\NormalTok{ (df)}
\NormalTok{\}}
  

\CommentTok{# add lagging data to the dataframe...}
\CommentTok{# the price we need to pay is loose one year of data for the predictions...}
\NormalTok{df_rm_huizenprijzen <-}\StringTok{ }
\StringTok{  }\NormalTok{df_rm_huizenprijzen }\OperatorTok
\StringTok{  }\KeywordTok{select}\NormalTok{(date, }
\NormalTok{         huisprijsindex_yoy,}
\NormalTok{         werkgelegenheid_yoy,}
\NormalTok{         hypotheek_rente_yoy,}
\NormalTok{         gdp,}
\NormalTok{         consumer_conf_yoy,}
\NormalTok{         inkomen_yoy) }\OperatorTok\StringTok{ }
\StringTok{  }\KeywordTok{lagdata}\NormalTok{() }\OperatorTok\StringTok{ }
\StringTok{  }\KeywordTok{drop_na}\NormalTok{()}
\end{Highlighting}
\end{Shaded}

\begin{Shaded}
\begin{Highlighting}[]
\CommentTok{# Split data set in Training and Testing set.}
\CommentTok{# Ideally we should use time slices and repeat the training for each period}
\CommentTok{# This type of analyses is beyond the scope of this research}
\NormalTok{df_train <-}\StringTok{ }\NormalTok{df_rm_huizenprijzen }\OperatorTok\StringTok{ }
\StringTok{  }\KeywordTok{filter}\NormalTok{(date }\OperatorTok{<=}\StringTok{ }\NormalTok{(}\KeywordTok{max}\NormalTok{(df_rm_huizenprijzen}\OperatorTok{$}\NormalTok{date) }\OperatorTok{-}\StringTok{ }\KeywordTok{years}\NormalTok{(}\DecValTok{5}\NormalTok{)))}
\NormalTok{df_test <-}\StringTok{ }\NormalTok{df_rm_huizenprijzen }\OperatorTok\StringTok{ }
\StringTok{  }\KeywordTok{filter}\NormalTok{(date }\OperatorTok{>}\StringTok{ }\NormalTok{(}\KeywordTok{max}\NormalTok{(df_rm_huizenprijzen}\OperatorTok{$}\NormalTok{date) }\OperatorTok{-}\StringTok{ }\KeywordTok{years}\NormalTok{(}\DecValTok{5}\NormalTok{)))}
\end{Highlighting}
\end{Shaded}

Lineair regressie modellen voor de verschillende indicatoren laten zien
dat BBP de beste fit geeft voor de huizenprijzen. Maar de onderlinge
verschillen zijn klein. De residuen na deze correlatie geven geen
duidelijk patroon aan. Het lijkt daarmee alsof dit de beste enkelvoudige
voorspeller is van de huisprijzen.

Met behulp van de caret library (Kuhn, 2020) kan een optimalisatie
worden uitgevoerd om het best passende model te selecteren. We zien hier
dat het model gebruik maakt van alle beschikbare data, en dan komt tot
een veel betere fit op de aangeboden training data.

\begin{Shaded}
\begin{Highlighting}[]
\CommentTok{# lineair regression models}

\NormalTok{lm_inkomen         =}\StringTok{ }\KeywordTok{lm}\NormalTok{(huisprijsindex_yoy}\OperatorTok{~}\NormalTok{inkomen_yoy, }\DataTypeTok{data =}\NormalTok{ df_train) }
\NormalTok{lm_time            =}\StringTok{ }\KeywordTok{lm}\NormalTok{(huisprijsindex_yoy }\OperatorTok{~}\StringTok{ }\NormalTok{date, }\DataTypeTok{data =}\NormalTok{ df_train)}
\NormalTok{lm_werkgelegenheid =}\StringTok{ }\KeywordTok{lm}\NormalTok{(huisprijsindex_yoy}\OperatorTok{~}\NormalTok{werkgelegenheid_yoy, }\DataTypeTok{data =}\NormalTok{ df_train) }
\NormalTok{lm_gdp             =}\StringTok{ }\KeywordTok{lm}\NormalTok{(huisprijsindex_yoy}\OperatorTok{~}\NormalTok{gdp, }\DataTypeTok{data =}\NormalTok{ df_train) }
\NormalTok{lm_rente           =}\StringTok{ }\KeywordTok{lm}\NormalTok{(huisprijsindex_yoy}\OperatorTok{~}\NormalTok{hypotheek_rente_yoy, }\DataTypeTok{data =}\NormalTok{ df_train) }
\NormalTok{lm_confidence      =}\StringTok{ }\KeywordTok{lm}\NormalTok{(huisprijsindex_yoy}\OperatorTok{~}\NormalTok{consumer_conf_yoy, }\DataTypeTok{data =}\NormalTok{ df_train) }

\KeywordTok{tab_model}\NormalTok{(lm_inkomen,lm_werkgelegenheid,lm_rente,}
          \DataTypeTok{use.viewer =} \OtherTok{TRUE}\NormalTok{,}
          \DataTypeTok{title =} \StringTok{'Vergelijking linaire regressie modellen'}\NormalTok{) }
\end{Highlighting}
\end{Shaded}

Vergelijking linaire regressie modellen

~

huisprijsindex yoy

huisprijsindex yoy

huisprijsindex yoy

Predictors

Estimates

CI

p

Estimates

CI

p

Estimates

CI

p

(Intercept)

0.64

-1.26~--~2.55

0.505

1.98

0.56~--~3.40

0.007

5.08

3.88~--~6.29

\textless0.001

inkomen\_yoy

1.69

1.07~--~2.32

\textless0.001

werkgelegenheid\_yoy

2.40

1.61~--~3.19

\textless0.001

hypotheek\_rente\_yoy

0.06

-0.03~--~0.16

0.211

Observations

111

111

111

R2 / R2 adjusted

0.209 / 0.202

0.250 / 0.244

0.014 / 0.005

\begin{Shaded}
\begin{Highlighting}[]
\KeywordTok{tab_model}\NormalTok{(lm_gdp,lm_time, lm_confidence,}
          \DataTypeTok{use.viewer =} \OtherTok{TRUE}\NormalTok{,}
          \DataTypeTok{title =} \StringTok{'Vergelijking linaire regressie modellen'}\NormalTok{)}
\end{Highlighting}
\end{Shaded}

Vergelijking linaire regressie modellen

~

huisprijsindex yoy

huisprijsindex yoy

huisprijsindex yoy

Predictors

Estimates

CI

p

Estimates

CI

p

Estimates

CI

p

(Intercept)

0.69

-0.72~--~2.10

0.336

18.20

14.03~--~22.36

\textless0.001

4.95

3.75~--~6.14

\textless0.001

gdp

1.92

1.45~--~2.39

\textless0.001

date

-0.00

-0.00~--~-0.00

\textless0.001

consumer\_conf\_yoy

-2.73

-25.71~--~20.25

0.814

Observations

111

111

111

R2 / R2 adjusted

0.372 / 0.367

0.279 / 0.273

0.001 / -0.009

\begin{Shaded}
\begin{Highlighting}[]
\CommentTok{# GDP is best fit for lineair regression}
\NormalTok{df_rm_lm <-}\StringTok{ }\KeywordTok{augment}\NormalTok{(lm_gdp)}
\KeywordTok{ggplot}\NormalTok{(df_rm_lm, }\KeywordTok{aes}\NormalTok{(}\DataTypeTok{x =}\NormalTok{ .fitted, }\DataTypeTok{y =}\NormalTok{ .resid)) }\OperatorTok{+}\StringTok{ }
\StringTok{  }\KeywordTok{geom_point}\NormalTok{(}\DataTypeTok{color =} \StringTok{'steelblue'}\NormalTok{) }\OperatorTok{+}
\StringTok{  }\KeywordTok{labs}\NormalTok{ (}
      \DataTypeTok{title =} \StringTok{'Residuals verband huizenprijzen en BBP'}\NormalTok{,}
      \DataTypeTok{x =} \StringTok{'lm(Huizenprijzen~BBP)'}\NormalTok{,}
      \DataTypeTok{y =} \StringTok{'Residuals'}
\NormalTok{    )}
\end{Highlighting}
\end{Shaded}

\includegraphics{data_exploration_files/figure-latex/regression_houseprices_lm-1.pdf}

\begin{Shaded}
\begin{Highlighting}[]
\CommentTok{# simple multiple regression model including all variables at T0}
\NormalTok{lm_multi =}\StringTok{ }\KeywordTok{lm}\NormalTok{(huisprijsindex_yoy}\OperatorTok{~}\NormalTok{date}\OperatorTok{+}
\StringTok{                }\NormalTok{werkgelegenheid_yoy}\OperatorTok{+}
\StringTok{                }\NormalTok{hypotheek_rente_yoy}\OperatorTok{+}
\StringTok{                }\NormalTok{gdp}\OperatorTok{+}
\StringTok{                }\NormalTok{consumer_conf_yoy}\OperatorTok{+}
\StringTok{                }\NormalTok{inkomen_yoy, }\DataTypeTok{data =}\NormalTok{ df_rm_huizenprijzen )}

\KeywordTok{tab_model}\NormalTok{(lm_multi,}
          \DataTypeTok{title =} \StringTok{'Multple regressie model'}\NormalTok{,}
          \DataTypeTok{use.viewer =} \OtherTok{TRUE}\NormalTok{)}
\end{Highlighting}
\end{Shaded}

Multple regressie model

~

huisprijsindex yoy

Predictors

Estimates

CI

p

(Intercept)

-1.53

-5.62~--~2.57

0.462

date

-0.00

-0.00~--~0.00

0.549

werkgelegenheid\_yoy

1.29

0.48~--~2.11

0.002

hypotheek\_rente\_yoy

-0.12

-0.20~--~-0.05

0.001

gdp

1.04

0.52~--~1.55

\textless0.001

consumer\_conf\_yoy

12.45

-1.93~--~26.82

0.089

inkomen\_yoy

1.42

0.94~--~1.90

\textless0.001

Observations

131

R2 / R2 adjusted

0.476 / 0.451

\begin{Shaded}
\begin{Highlighting}[]
\CommentTok{# The carot model can train the best multiple regression model optimizing RMSE (Root Mean Squared Error).}
\NormalTok{lm_compute <-}\StringTok{ }\KeywordTok{train}\NormalTok{(}
\NormalTok{  huisprijsindex_yoy }\OperatorTok{~}\StringTok{ }\NormalTok{., }
  \DataTypeTok{data =}\NormalTok{ df_train, }
  \DataTypeTok{method =} \StringTok{"lm"}\NormalTok{,}
  \DataTypeTok{metric =} \StringTok{"RMSE"}\NormalTok{,}
  \DataTypeTok{trControl =} \KeywordTok{trainControl}\NormalTok{(}\DataTypeTok{method =} \StringTok{"cv"}\NormalTok{, }\DataTypeTok{number =} \DecValTok{10}\NormalTok{),}
  \DataTypeTok{preProcess =} \KeywordTok{c}\NormalTok{(}\StringTok{"zv"}\NormalTok{, }\StringTok{"center"}\NormalTok{, }\StringTok{"scale"}\NormalTok{)}
\NormalTok{  )}

\KeywordTok{tab_model}\NormalTok{(lm_compute}\OperatorTok{$}\NormalTok{finalModel,}
          \DataTypeTok{title =} \StringTok{"Time Series Lineair Model"}\NormalTok{,}
          \DataTypeTok{use.viewer =} \OtherTok{TRUE}
\NormalTok{          )}
\end{Highlighting}
\end{Shaded}

Time Series Lineair Model

~

outcome

Predictors

Estimates

CI

p

(Intercept)

4.94

4.66~--~5.21

\textless0.001

date

0.00

-0.40~--~0.40

0.986

werkgelegenheid\_yoy

-0.53

-2.13~--~1.08

0.516

hypotheek\_rente\_yoy

-0.14

-0.92~--~0.64

0.721

gdp

0.57

-0.34~--~1.47

0.217

consumer\_conf\_yoy

0.19

-0.11~--~0.49

0.202

inkomen\_yoy

0.30

-0.36~--~0.96

0.371

huisprijsindex\_yoy1

5.14

3.81~--~6.47

\textless0.001

huisprijsindex\_yoy2

0.02

-1.68~--~1.71

0.986

huisprijsindex\_yoy3

0.84

-0.88~--~2.57

0.333

huisprijsindex\_yoy4

-1.35

-2.50~--~-0.20

0.022

werkgelegenheid\_yoy1

0.10

-2.35~--~2.55

0.935

werkgelegenheid\_yoy2

-0.82

-3.38~--~1.73

0.522

werkgelegenheid\_yoy3

0.13

-2.31~--~2.57

0.917

werkgelegenheid\_yoy4

1.22

-0.22~--~2.65

0.095

hypotheek\_rente\_yoy1

-1.04

-2.29~--~0.20

0.100

hypotheek\_rente\_yoy2

0.09

-1.23~--~1.41

0.888

hypotheek\_rente\_yoy3

-0.09

-1.31~--~1.13

0.885

hypotheek\_rente\_yoy4

0.14

-0.66~--~0.94

0.730

gdp1

1.44

0.20~--~2.69

0.024

gdp2

0.01

-1.22~--~1.23

0.991

gdp3

-0.53

-1.73~--~0.66

0.376

gdp4

0.65

-0.31~--~1.60

0.181

consumer\_conf\_yoy1

0.07

-0.23~--~0.37

0.626

consumer\_conf\_yoy2

-0.09

-0.39~--~0.21

0.552

consumer\_conf\_yoy3

-0.19

-0.49~--~0.10

0.196

consumer\_conf\_yoy4

-0.28

-0.57~--~0.01

0.061

inkomen\_yoy1

-0.61

-1.41~--~0.18

0.130

inkomen\_yoy2

-0.07

-0.88~--~0.74

0.860

inkomen\_yoy3

0.28

-0.53~--~1.08

0.496

inkomen\_yoy4

0.45

-0.26~--~1.15

0.210

Observations

111

R2 / R2 adjusted

0.962 / 0.948

Tot slot is gekeken of nog een beter passend model verkregen kan worden
met MARS (multivariate adaptive regression splines {[}Friedman (1991a,
1991b){]}. Deze methode splits de regressie in meerdere `splines'.
Waardoor in theorie een betere fit gevonden kan worden door rekening te
houden met een non-lineaire relatie over de tijd.

Het MARS model kiest zelf de meest relevante parameters. Wanneer met dit
model geprobeerd wordt om de prijsontwikkeling van de afgelopen 5 jaar
te voorspellen geeft dit model een goede fit.

\begin{Shaded}
\begin{Highlighting}[]
\CommentTok{# Simple MARS model}

\CommentTok{# Tuning the model}
\CommentTok{# Optimal number of defrees and number of variables to prune is calculated}

\CommentTok{# create a tuning grid }
\NormalTok{hyper_grid <-}\StringTok{ }\KeywordTok{expand.grid}\NormalTok{(}
  \DataTypeTok{degree =} \DecValTok{1}\OperatorTok{:}\DecValTok{3}\NormalTok{, }
  \DataTypeTok{nprune =} \KeywordTok{seq}\NormalTok{(}\DecValTok{2}\NormalTok{, }\DecValTok{100}\NormalTok{, }\DataTypeTok{length.out =} \DecValTok{20}\NormalTok{) }\OperatorTok\StringTok{ }\KeywordTok{floor}\NormalTok{()}
\NormalTok{  )}

\CommentTok{# for reproducibiity}
\KeywordTok{set.seed}\NormalTok{(}\DecValTok{123}\NormalTok{)}

\CommentTok{# cross validated model}
\NormalTok{tuned_mars <-}\StringTok{ }\KeywordTok{train}\NormalTok{(}
  \DataTypeTok{x =} \KeywordTok{subset}\NormalTok{(df_train, }\DataTypeTok{select =} \OperatorTok{-}\NormalTok{huisprijsindex_yoy),}
  \DataTypeTok{y =}\NormalTok{ df_train}\OperatorTok{$}\NormalTok{huisprijsindex_yoy,}
  \DataTypeTok{method =} \StringTok{"earth"}\NormalTok{,}
  \DataTypeTok{metric =} \StringTok{"RMSE"}\NormalTok{,}
  \DataTypeTok{trControl =} \KeywordTok{trainControl}\NormalTok{(}\DataTypeTok{method =} \StringTok{"cv"}\NormalTok{, }\DataTypeTok{number =} \DecValTok{10}\NormalTok{),}
  \DataTypeTok{tuneGrid =}\NormalTok{ hyper_grid}
\NormalTok{)}

\CommentTok{# plot results}
\CommentTok{# I was not able to change the legend title}
\KeywordTok{ggplot}\NormalTok{(tuned_mars) }\OperatorTok{+}
\StringTok{  }\KeywordTok{labs}\NormalTok{ (}
      \DataTypeTok{title =} \StringTok{'Analyse voorspelling factoren MARS model'}\NormalTok{,}
      \DataTypeTok{x =} \StringTok{'Aantal factoren'}\NormalTok{,}
      \DataTypeTok{y =} \StringTok{'Gemiddelde kwadratischfout (RMSE)'}
\NormalTok{    ) }
\end{Highlighting}
\end{Shaded}

\includegraphics{data_exploration_files/figure-latex/unnamed-chunk-1-1.pdf}

Door middel van de optimalisatie wordt hier het aantal factoren en de
mate van interactie tussen de factoren (product degree) geselecteerd. De
belangrijkste factoren staan hieronder genoemd.

\begin{Shaded}
\begin{Highlighting}[]
\CommentTok{# fit model}
\NormalTok{fit <-}\StringTok{ }\KeywordTok{earth}\NormalTok{(huisprijsindex_yoy}\OperatorTok{~}\NormalTok{., df_train, }
             \DataTypeTok{degree =}\NormalTok{ tuned_mars}\OperatorTok{$}\NormalTok{bestTune}\OperatorTok{$}\NormalTok{degree, }\DataTypeTok{nprune =}\NormalTok{ tuned_mars}\OperatorTok{$}\NormalTok{bestTune}\OperatorTok{$}\NormalTok{nprune)}

\CommentTok{# summarize the importance of input variables}
\KeywordTok{kable}\NormalTok{(}\KeywordTok{print}\NormalTok{(}\KeywordTok{evimp}\NormalTok{(fit))}
\NormalTok{      )}
\end{Highlighting}
\end{Shaded}

\begin{verbatim}
##                      nsubsets   gcv    rss
## huisprijsindex_yoy1         6 100.0  100.0
## hypotheek_rente_yoy3        4  14.4   17.6
## gdp4                        4  14.4   17.6
## huisprijsindex_yoy4         3   9.8   13.2
## werkgelegenheid_yoy4        2   9.2   11.3
## hypotheek_rente_yoy1        1   8.3    9.1
\end{verbatim}

\begin{tabular}{}
\hline

\hline
\end{tabular}

\begin{Shaded}
\begin{Highlighting}[]
\CommentTok{# make predictions for MARS model}
\NormalTok{train_predictions <-}\StringTok{ }\KeywordTok{predict}\NormalTok{(fit, df_train)}
\NormalTok{test_predictions <-}\StringTok{ }\KeywordTok{predict}\NormalTok{(fit, df_test)}

\CommentTok{# make predictions for our regression model}
\NormalTok{train_lm_predictions <-}\StringTok{ }\KeywordTok{predict}\NormalTok{(lm_multi, df_train)}
\NormalTok{test_lm_predictions <-}\StringTok{ }\KeywordTok{predict}\NormalTok{(lm_multi, df_test)}

\CommentTok{# make predictions for our ts regression model}
\NormalTok{train_lmts_predictions <-}\StringTok{ }\KeywordTok{predict}\NormalTok{(lm_compute, df_train)}
\NormalTok{test_lmts_predictions <-}\StringTok{ }\KeywordTok{predict}\NormalTok{(lm_compute, df_test)}

\NormalTok{df_result <-}
\StringTok{  }\KeywordTok{tibble}\NormalTok{ (}
    \DataTypeTok{date =}\NormalTok{ df_train}\OperatorTok{$}\NormalTok{date, }
    \DataTypeTok{huisprijsindex_yoy =}\NormalTok{ df_train}\OperatorTok{$}\NormalTok{huisprijsindex_yoy,  }
    \DataTypeTok{predicted =}\NormalTok{ train_predictions,}
    \DataTypeTok{variable =} \StringTok{'train'}\NormalTok{,}
    \DataTypeTok{model =} \StringTok{'MARS'}
\NormalTok{  ) }\OperatorTok\StringTok{ }
\StringTok{  }\KeywordTok{union_all}\NormalTok{( }
    \KeywordTok{tibble}\NormalTok{(}
      \DataTypeTok{date =}\NormalTok{ df_test}\OperatorTok{$}\NormalTok{date, }
      \DataTypeTok{huisprijsindex_yoy =}\NormalTok{ df_test}\OperatorTok{$}\NormalTok{huisprijsindex_yoy,  }
      \DataTypeTok{predicted =}\NormalTok{ test_predictions,}
      \DataTypeTok{variable =} \StringTok{'test'}\NormalTok{,}
      \DataTypeTok{model =} \StringTok{'MARS'}\NormalTok{)) }\OperatorTok\StringTok{ }
\StringTok{  }\KeywordTok{union_all}\NormalTok{( }
    \KeywordTok{tibble}\NormalTok{(}
      \DataTypeTok{date =}\NormalTok{ df_train}\OperatorTok{$}\NormalTok{date, }
      \DataTypeTok{huisprijsindex_yoy =}\NormalTok{ df_train}\OperatorTok{$}\NormalTok{huisprijsindex_yoy,  }
      \DataTypeTok{predicted =}\NormalTok{ train_lm_predictions,}
      \DataTypeTok{variable =} \StringTok{'train'}\NormalTok{,}
      \DataTypeTok{model =} \StringTok{'LM'}\NormalTok{)) }\OperatorTok\StringTok{ }
\StringTok{  }\KeywordTok{union_all}\NormalTok{( }
    \KeywordTok{tibble}\NormalTok{(}
      \DataTypeTok{date =}\NormalTok{ df_test}\OperatorTok{$}\NormalTok{date, }
      \DataTypeTok{huisprijsindex_yoy =}\NormalTok{ df_test}\OperatorTok{$}\NormalTok{huisprijsindex_yoy,  }
      \DataTypeTok{predicted =}\NormalTok{ test_lm_predictions,}
      \DataTypeTok{variable =} \StringTok{'test'}\NormalTok{,}
      \DataTypeTok{model =} \StringTok{'LM'}\NormalTok{)) }\OperatorTok\StringTok{ }
\StringTok{  }\KeywordTok{union_all}\NormalTok{( }
    \KeywordTok{tibble}\NormalTok{(}
      \DataTypeTok{date =}\NormalTok{ df_train}\OperatorTok{$}\NormalTok{date, }
      \DataTypeTok{huisprijsindex_yoy =}\NormalTok{ df_train}\OperatorTok{$}\NormalTok{huisprijsindex_yoy,  }
      \DataTypeTok{predicted =}\NormalTok{ train_lmts_predictions,}
      \DataTypeTok{variable =} \StringTok{'train'}\NormalTok{,}
      \DataTypeTok{model =} \StringTok{'LMTS'}\NormalTok{) }\OperatorTok\StringTok{ }
\StringTok{  }\KeywordTok{union_all}\NormalTok{( }
    \KeywordTok{tibble}\NormalTok{(}
      \DataTypeTok{date =}\NormalTok{ df_test}\OperatorTok{$}\NormalTok{date, }
      \DataTypeTok{huisprijsindex_yoy =}\NormalTok{ df_test}\OperatorTok{$}\NormalTok{huisprijsindex_yoy,  }
      \DataTypeTok{predicted =}\NormalTok{ test_lmts_predictions,}
      \DataTypeTok{variable =} \StringTok{'test'}\NormalTok{,}
      \DataTypeTok{model =} \StringTok{'LMTS'}\NormalTok{))}
\NormalTok{  )}
  
\NormalTok{df_result }\OperatorTok\StringTok{ }
\StringTok{  }\KeywordTok{ggplot}\NormalTok{(}\KeywordTok{aes}\NormalTok{(date, huisprijsindex_yoy)) }\OperatorTok{+}
\StringTok{  }\KeywordTok{geom_line}\NormalTok{(}\DataTypeTok{size =} \FloatTok{.5}\NormalTok{) }\OperatorTok{+}
\StringTok{  }\KeywordTok{geom_line}\NormalTok{(}\KeywordTok{aes}\NormalTok{(}\DataTypeTok{y =}\NormalTok{ predicted, }\DataTypeTok{color =}\NormalTok{ model, }\DataTypeTok{linetype =}\NormalTok{ variable), }
            \DataTypeTok{size =} \DecValTok{1}\NormalTok{, }\DataTypeTok{alpha =} \FloatTok{0.8}\NormalTok{) }\OperatorTok{+}
\StringTok{  }\KeywordTok{scale_linetype_manual}\NormalTok{(}\DataTypeTok{values=}\KeywordTok{c}\NormalTok{(}\StringTok{"dotted"}\NormalTok{,}\StringTok{"solid"}\NormalTok{))}\OperatorTok{+}
\StringTok{  }\KeywordTok{labs}\NormalTok{(}
    \DataTypeTok{title =} \StringTok{"Voorspelling huizenprijzen MARS Model"}\NormalTok{,}
    \DataTypeTok{y =} \StringTok{"huisprijsindex"}\NormalTok{,}
    \DataTypeTok{x =} \StringTok{''}\NormalTok{,}
    \DataTypeTok{color =} \StringTok{'dataset'}
\NormalTok{  ) }\OperatorTok{+}
\StringTok{  }\KeywordTok{theme}\NormalTok{(}\DataTypeTok{legend.position=}\StringTok{"bottom"}\NormalTok{)}
\end{Highlighting}
\end{Shaded}

\includegraphics{data_exploration_files/figure-latex/regression_houseprices_mars-1.pdf}

In dit geval geeft het lineair regressie model duidelijk een minder
goede fit dan het time series model en het MARS model. Het MARS model
sluit heel goed aan bij de ontwikkeling van de huizenprijzen over de
afgelopen vijf jaar.

\hypertarget{regionale-verschillen}{%
\subsubsection{Regionale verschillen}\label{regionale-verschillen}}

Er bestaan grote regionale verschillen in huisprijzen, en in de
prijsontwikkeling over de tijd.

\begin{Shaded}
\begin{Highlighting}[]
\CommentTok{# boxplot}
\NormalTok{verkoopprijs_regio }\OperatorTok\StringTok{ }
\StringTok{  }\KeywordTok{filter}\NormalTok{(date }\OperatorTok{>=}\StringTok{ }\KeywordTok{as.Date}\NormalTok{(}\StringTok{'1995-01-01'}\NormalTok{)}
\NormalTok{             ) }\OperatorTok\StringTok{ }
\StringTok{  }\KeywordTok{drop_na}\NormalTok{(verkoopprijs) }\OperatorTok\StringTok{ }
\StringTok{  }\KeywordTok{mutate}\NormalTok{(}\DataTypeTok{verkoopprijs =}\NormalTok{ verkoopprijs }\OperatorTok{/}\StringTok{ }\DecValTok{1000}\NormalTok{) }\OperatorTok\StringTok{ }
\StringTok{  }\KeywordTok{ggplot}\NormalTok{(}\KeywordTok{aes}\NormalTok{(}\DataTypeTok{y =}\NormalTok{ verkoopprijs, }\DataTypeTok{x =}\NormalTok{ jaar, }\DataTypeTok{fill =}\NormalTok{ jaar) ) }\OperatorTok{+}
\StringTok{  }\KeywordTok{geom_boxplot}\NormalTok{() }\OperatorTok{+}
\StringTok{  }\KeywordTok{labs}\NormalTok{(}\DataTypeTok{title =} \StringTok{"Distributie verkoopprijzen"}\NormalTok{,}
       \DataTypeTok{subtitle =} \StringTok{"gemiddelde verkoopprijs per gemeente"}\NormalTok{,}
        \DataTypeTok{x =} \StringTok{"Verkoopprijs (x 1000 eur)"}\NormalTok{) }\OperatorTok{+}
\StringTok{  }\KeywordTok{theme}\NormalTok{(}\DataTypeTok{legend.position =} \StringTok{"none"}\NormalTok{) }\OperatorTok{+}
\StringTok{  }\KeywordTok{theme}\NormalTok{(}\DataTypeTok{axis.text.x =} \KeywordTok{element_text}\NormalTok{(}\DataTypeTok{angle =} \DecValTok{60}\NormalTok{, }\DataTypeTok{hjust =} \DecValTok{1}\NormalTok{))}
\end{Highlighting}
\end{Shaded}

\includegraphics{data_exploration_files/figure-latex/regional_sales-1.pdf}

\begin{Shaded}
\begin{Highlighting}[]
\CommentTok{# Create a thematic map}
\NormalTok{verkoopprijs_regio }\OperatorTok
\StringTok{  }\KeywordTok{filter}\NormalTok{(date }\OperatorTok{>=}\StringTok{ }\KeywordTok{as.Date}\NormalTok{(}\StringTok{'2015-01-01'}\NormalTok{)}
\NormalTok{           ) }\OperatorTok\StringTok{ }
\StringTok{  }\KeywordTok{mutate}\NormalTok{(}\DataTypeTok{verkoopprijs =}\NormalTok{ verkoopprijs }\OperatorTok{/}\StringTok{ }\DecValTok{1000}\NormalTok{) }\OperatorTok\StringTok{ }
\StringTok{  }\KeywordTok{ggplot}\NormalTok{() }\OperatorTok{+}
\StringTok{  }\KeywordTok{geom_sf}\NormalTok{(}\KeywordTok{aes}\NormalTok{(}\DataTypeTok{fill =}\NormalTok{ verkoopprijs)) }\OperatorTok{+}
\StringTok{  }\KeywordTok{facet_wrap}\NormalTok{(}\OperatorTok{~}\NormalTok{jaar) }\OperatorTok{+}
\StringTok{  }\KeywordTok{scale_fill_viridis_c}\NormalTok{(}\DataTypeTok{option =} \StringTok{"inferno"}\NormalTok{) }\OperatorTok{+}
\StringTok{  }\KeywordTok{labs}\NormalTok{(}\DataTypeTok{title =} \StringTok{"Gemiddelde verkoopprijzen woningen"}\NormalTok{, }
       \DataTypeTok{subtitle =} \StringTok{""}\NormalTok{,}
       \DataTypeTok{fill =} \StringTok{"(x 1000 eur)"}\NormalTok{) }\OperatorTok{+}
\StringTok{  }\KeywordTok{theme_void}\NormalTok{()}
\end{Highlighting}
\end{Shaded}

\includegraphics{data_exploration_files/figure-latex/regional_sales-2.pdf}

\hypertarget{aantal-transacties}{%
\subsection{Aantal transacties}\label{aantal-transacties}}

Het aantal transacties heeft in de crisis jaren 2009 - 2012 een
duidelijke terugval laten zien. Sindsdien zien we een snelle
inhaalbeweging, en nog steeds zien we aanzienlijk hogere transactie
volumes dan voorheen.

\begin{Shaded}
\begin{Highlighting}[]
\NormalTok{verkochte_woningen_per_type_per_kwartaal }\OperatorTok\StringTok{ }
\StringTok{  }\KeywordTok{mutate}\NormalTok{(}\DataTypeTok{subtype =} \KeywordTok{factor}\NormalTok{(subtype, }\DataTypeTok{levels=}\KeywordTok{c}\NormalTok{(}\StringTok{'Vrijstaande woning'}\NormalTok{,}
                                  \StringTok{'2-onder-1-kapwoning'}\NormalTok{,}
                                  \StringTok{'Hoekwoning'}\NormalTok{,}
                                  \StringTok{'Tussenwoning'}\NormalTok{,}
                                  \StringTok{'Appartement'}\NormalTok{,}
                                  \StringTok{'Onbekend'}\NormalTok{))) }\OperatorTok\StringTok{ }
\StringTok{  }\KeywordTok{ggplot}\NormalTok{(}\KeywordTok{aes}\NormalTok{(}\DataTypeTok{x=}\NormalTok{date, }\DataTypeTok{y=}\NormalTok{verkochte_bestaande_woningen )) }\OperatorTok{+}
\StringTok{  }\KeywordTok{geom_line}\NormalTok{(}\KeywordTok{aes}\NormalTok{(}\DataTypeTok{color =}\NormalTok{ subtype)) }\OperatorTok{+}
\StringTok{  }\KeywordTok{facet_wrap}\NormalTok{(}\OperatorTok{~}\StringTok{ }\NormalTok{subtype, }\DataTypeTok{ncol =} \DecValTok{2}\NormalTok{, }\DataTypeTok{scales =} \StringTok{"free"}\NormalTok{) }\OperatorTok{+}
\StringTok{  }\KeywordTok{labs}\NormalTok{(}
    \DataTypeTok{title =} \StringTok{"Verkopen Bestaande Koopwoningen per kwartaal"}\NormalTok{,}
    \DataTypeTok{y =} \StringTok{""}\NormalTok{,}
    \DataTypeTok{x =} \StringTok{''}\NormalTok{,}
    \DataTypeTok{color =} \StringTok{'subtype'}
\NormalTok{    ) }\OperatorTok{+}
\StringTok{  }\KeywordTok{theme}\NormalTok{(}\DataTypeTok{legend.position =} \StringTok{"none"}\NormalTok{) }
\end{Highlighting}
\end{Shaded}

\includegraphics{data_exploration_files/figure-latex/sales_per_q-1.pdf}

Wanneer gekeken wordt naar de transactie volumes valt de piek op in
appartementen in het laatste kwartaal van 2020. Dit is te verklaren door
de wijziging in de regelgeving voor beleggers. Particuliere beleggers op
de woningmarkt moeten vanaf 2021 8\% overdrachtsbelasting betalen. Dit
heeft gezorgd voor een eenmalige piek in aankopen door deze groep.
Tegelijkertijd is er begin 2021 een vrijstelling van
overdrachtsbelasting voor jongeren tot 35 jaar. Jongeren hebben daarom
veelal de aankoop uitgesteld naar het eerste kwartaal van 2021. Deze
effecten van regelgeving zien we terug in de data.

Voorspellen van aantal transacties

Ook hier wordt eerst gekeken naar de correlaties tussen de variabelen.
Wederom is de data omgerekend naar jaar-op-jaar percentuele wijzigingen
om de factoren vergelijkbaar te maken. Het aantal transacties is slechts
heel beperkt gecorreleerd aan de geselecteerde variabelen. Opmerkelijk
is dat we bijna geen correlatie zien tussen het aantal transacties en
het aantal huishoudens.

\begin{Shaded}
\begin{Highlighting}[]
\NormalTok{corMatrix <-}\StringTok{ }\KeywordTok{round}\NormalTok{(}\KeywordTok{cor}\NormalTok{(df_rm_transactions }\OperatorTok\StringTok{ }
\StringTok{                         }\KeywordTok{select}\NormalTok{(}\OperatorTok{-}\NormalTok{date)),}\DecValTok{5}\NormalTok{)}

\NormalTok{corCols <-}\StringTok{ }\KeywordTok{c}\NormalTok{(}\StringTok{"huisprijsindex"}\NormalTok{, }
             \StringTok{"Inkomen"}\NormalTok{,}
             \StringTok{"werkgelegenheid"}\NormalTok{,}
             \StringTok{"BBP"}\NormalTok{,}
             \StringTok{"Cons.Vertrouwen"}\NormalTok{, }
             \StringTok{"Aantal huishoudens"}\NormalTok{,}
             \StringTok{"Aantal transacties"}\NormalTok{,}
             \StringTok{"Woning tekort"}\NormalTok{)}

\KeywordTok{colnames}\NormalTok{(corMatrix) <-}\StringTok{ }\NormalTok{corCols}
\KeywordTok{rownames}\NormalTok{(corMatrix) <-}\StringTok{ }\NormalTok{corCols}

\KeywordTok{corrplot}\NormalTok{(corMatrix,}
            \DataTypeTok{title =} \StringTok{"Correlatie transacties"}\NormalTok{,}
            \DataTypeTok{type =} \StringTok{"lower"}\NormalTok{, }
            \DataTypeTok{tl.cex =} \FloatTok{0.8}\NormalTok{,}
            \DataTypeTok{tl.col =} \StringTok{"black"}\NormalTok{, }
            \DataTypeTok{tl.srt =} \DecValTok{45}\NormalTok{,}
            \DataTypeTok{mar=}\KeywordTok{c}\NormalTok{(}\DecValTok{0}\NormalTok{, }\DecValTok{0}\NormalTok{, }\DecValTok{4}\NormalTok{, }\DecValTok{0}\NormalTok{)) }
\end{Highlighting}
\end{Shaded}

\includegraphics{data_exploration_files/figure-latex/sales_correlation-1.pdf}

Voorspellen aantal transacties

Ook is het mogelijk om net als bij de huizenprijzen, een MARS model toe
te passen op het aantal transacties. Het MARS model selecteert als de
belangrijkste voorspellers: aantal huishoudens, BBP, tijdsfactor (het
aantal transacties in de vorige periode) en het consumentenvertrouwen.

Het MARS model geeft op basis van deze gegevens geen goede voorspelling.
Ergens gaat het mis met de data in 2020. Mogelijk dat de effecten van de
COVID-crisis hier onvoorziene invloeden hebben op de voorspelling.

\begin{Shaded}
\begin{Highlighting}[]
\CommentTok{# Add Lineair regression model - for number of households and transactions...}

\CommentTok{# Add 4 lags of data}
\NormalTok{df_rm_transactions <-}
\StringTok{  }\NormalTok{df_rm_transactions }\OperatorTok\StringTok{ }
\StringTok{  }\KeywordTok{lagdata}\NormalTok{() }\OperatorTok\StringTok{ }
\StringTok{  }\KeywordTok{drop_na}\NormalTok{()}
  
\CommentTok{# Split in Train and Test data set}
\NormalTok{df_train <-}\StringTok{ }\NormalTok{df_rm_transactions }\OperatorTok\StringTok{ }
\StringTok{  }\KeywordTok{filter}\NormalTok{(date }\OperatorTok{<=}\StringTok{ }\NormalTok{(}\KeywordTok{max}\NormalTok{(df_rm_transactions}\OperatorTok{$}\NormalTok{date) }\OperatorTok{-}\StringTok{ }\KeywordTok{years}\NormalTok{(}\DecValTok{5}\NormalTok{)))}
\NormalTok{df_test <-}\StringTok{ }\NormalTok{df_rm_transactions }\OperatorTok\StringTok{ }
\StringTok{  }\KeywordTok{filter}\NormalTok{(date }\OperatorTok{>}\StringTok{ }\NormalTok{(}\KeywordTok{max}\NormalTok{(df_rm_transactions}\OperatorTok{$}\NormalTok{date) }\OperatorTok{-}\StringTok{ }\KeywordTok{years}\NormalTok{(}\DecValTok{5}\NormalTok{)))}


\CommentTok{# Tuning the model}
\CommentTok{# Optimal number of defrees and number of variables to prune is calculated}

\CommentTok{# create a tuning grid }
\NormalTok{hyper_grid <-}\StringTok{ }\KeywordTok{expand.grid}\NormalTok{(}
  \DataTypeTok{degree =} \DecValTok{1}\OperatorTok{:}\DecValTok{3}\NormalTok{, }
  \DataTypeTok{nprune =} \KeywordTok{seq}\NormalTok{(}\DecValTok{2}\NormalTok{, }\DecValTok{100}\NormalTok{, }\DataTypeTok{length.out =} \DecValTok{20}\NormalTok{) }\OperatorTok\StringTok{ }\KeywordTok{floor}\NormalTok{()}
\NormalTok{)}


\CommentTok{# for reproducibiity}
\KeywordTok{set.seed}\NormalTok{(}\DecValTok{123}\NormalTok{)}

\CommentTok{# cross validated model}
\NormalTok{tuned_mars <-}\StringTok{ }\KeywordTok{train}\NormalTok{(}
  \DataTypeTok{x =} \KeywordTok{subset}\NormalTok{(df_train, }\DataTypeTok{select =} \OperatorTok{-}\NormalTok{verkocht_yoy),}
  \DataTypeTok{y =}\NormalTok{ df_train}\OperatorTok{$}\NormalTok{verkocht_yoy,}
  \DataTypeTok{method =} \StringTok{"earth"}\NormalTok{,}
  \DataTypeTok{metric =} \StringTok{"RMSE"}\NormalTok{,}
  \DataTypeTok{trControl =} \KeywordTok{trainControl}\NormalTok{(}\DataTypeTok{method =} \StringTok{"cv"}\NormalTok{, }\DataTypeTok{number =} \DecValTok{10}\NormalTok{),}
  \DataTypeTok{tuneGrid =}\NormalTok{ hyper_grid}
\NormalTok{)}

\CommentTok{# plot results}
\KeywordTok{ggplot}\NormalTok{(tuned_mars) }\OperatorTok{+}
\StringTok{  }\KeywordTok{labs}\NormalTok{ (}
      \DataTypeTok{title =} \StringTok{'Analyse voorspelling factoren MARS model transactievolumes'}\NormalTok{,}
      \DataTypeTok{x =} \StringTok{'Aantal factoren'}\NormalTok{,}
      \DataTypeTok{y =} \StringTok{'Gemiddelde kwadratischfout (RMSE)'}
\NormalTok{    ) }
\end{Highlighting}
\end{Shaded}

\includegraphics{data_exploration_files/figure-latex/sales_mars-1.pdf}

\begin{Shaded}
\begin{Highlighting}[]
\CommentTok{# fit model}
\NormalTok{fit <-}\StringTok{ }\KeywordTok{earth}\NormalTok{(verkocht_yoy}\OperatorTok{~}\NormalTok{., df_train, }
             \DataTypeTok{degree =}\NormalTok{ tuned_mars}\OperatorTok{$}\NormalTok{bestTune}\OperatorTok{$}\NormalTok{degree, }\DataTypeTok{nprune =}\NormalTok{ tuned_mars}\OperatorTok{$}\NormalTok{bestTune}\OperatorTok{$}\NormalTok{nprune)}
\CommentTok{# summarize the fit}
\CommentTok{# summary(fit)}

\CommentTok{# summarize the importance of input variables}
\KeywordTok{kable}\NormalTok{(}\KeywordTok{print}\NormalTok{(}\KeywordTok{evimp}\NormalTok{(fit)))}
\end{Highlighting}
\end{Shaded}

\begin{verbatim}
##                      nsubsets   gcv    rss
## aantal_hh_yoy               6 100.0  100.0
## gdp                         5  76.0   76.4
## verkocht_yoy1               4  50.1   53.1
## verkocht_yoy3-unused        1  19.6   22.3
\end{verbatim}

\begin{tabular}{}
\hline

\hline
\end{tabular}

\begin{Shaded}
\begin{Highlighting}[]
\CommentTok{# make predictions}
\NormalTok{train_predictions <-}\StringTok{ }\KeywordTok{predict}\NormalTok{(fit, df_train)}
\NormalTok{test_predictions <-}\StringTok{ }\KeywordTok{predict}\NormalTok{(fit, df_test)}

\NormalTok{df_result <-}
\StringTok{  }\KeywordTok{tibble}\NormalTok{ (}
    \DataTypeTok{date =}\NormalTok{ df_train}\OperatorTok{$}\NormalTok{date, }
    \DataTypeTok{verkocht_yoy =}\NormalTok{ df_train}\OperatorTok{$}\NormalTok{verkocht_yoy,  }
    \DataTypeTok{predicted =}\NormalTok{ train_predictions,}
    \DataTypeTok{variable =} \StringTok{'train'}
\NormalTok{  ) }\OperatorTok\StringTok{ }
\StringTok{  }\KeywordTok{union_all}\NormalTok{( }
    \KeywordTok{tibble}\NormalTok{(}
      \DataTypeTok{date =}\NormalTok{ df_test}\OperatorTok{$}\NormalTok{date, }
      \DataTypeTok{verkocht_yoy =}\NormalTok{ df_test}\OperatorTok{$}\NormalTok{verkocht_yoy,  }
      \DataTypeTok{predicted =}\NormalTok{ test_predictions,}
      \DataTypeTok{variable =} \StringTok{'test'}\NormalTok{)}
\NormalTok{)}

\NormalTok{df_result }\OperatorTok\StringTok{ }
\StringTok{  }\KeywordTok{ggplot}\NormalTok{(}\KeywordTok{aes}\NormalTok{(date, verkocht_yoy)) }\OperatorTok{+}
\StringTok{  }\KeywordTok{geom_line}\NormalTok{(}\DataTypeTok{size =} \FloatTok{0.5}\NormalTok{) }\OperatorTok{+}
\StringTok{  }\KeywordTok{geom_line}\NormalTok{(}\KeywordTok{aes}\NormalTok{(}\DataTypeTok{y =}\NormalTok{ predicted, }\DataTypeTok{linetype =}\NormalTok{ variable, }\DataTypeTok{color =} \StringTok{'steelblue'}\NormalTok{), }\DataTypeTok{size =} \DecValTok{1}\NormalTok{) }\OperatorTok{+}
\StringTok{  }\KeywordTok{scale_linetype_manual}\NormalTok{(}\DataTypeTok{values=}\KeywordTok{c}\NormalTok{(}\StringTok{"dotted"}\NormalTok{,}\StringTok{"solid"}\NormalTok{)) }\OperatorTok{+}
\StringTok{  }\KeywordTok{scale_color_manual}\NormalTok{(}\DataTypeTok{values=}\KeywordTok{c}\NormalTok{(}\StringTok{"steelblue"}\NormalTok{)) }\OperatorTok{+}
\StringTok{  }\KeywordTok{labs}\NormalTok{(}
    \DataTypeTok{title =} \StringTok{"Voorspelling aantal transacties MARS Model"}\NormalTok{,}
    \DataTypeTok{y =} \StringTok{"Aantal transacties"}\NormalTok{,}
    \DataTypeTok{x =} \StringTok{''}\NormalTok{,}
    \DataTypeTok{linetype =} \StringTok{'dataset'}
\NormalTok{  ) }
\end{Highlighting}
\end{Shaded}

\includegraphics{data_exploration_files/figure-latex/sales_mars-2.pdf}

Seizoensinvloeden bij het aantal transacties

Om de ontwikkeling van de transactievolumes verder te analyseren is een
tijdreeksanalyse uitgevoerd op de data. De ACf (auto correlatie) plot
toont dat meer recente data significant is, wat een trend aangeeft in de
data. De PACF plot (partial auto-correlation function) toont dat recente
data en seizoensinvloeden een goede indicatie geven voor het modeleren
van de trend.

De uiteindelijke voorspelling laat het seizoenspatroon zien. Verder
wordt door dit model de recente trend doorgetrokken, terwijl eigenlijk
verwacht zou mogen worden dat op termijn de volumes op de huizenmarkt
terugkeren naar een stabiel niveau.

\begin{Shaded}
\begin{Highlighting}[]
\CommentTok{# }
\CommentTok{# vw <- verkochte_woningen_per_type_per_kwartaal %>% }
\CommentTok{#   group_by (yearqtr) %>% }
\CommentTok{#   summarise(totaal = sum(verkochte_bestaande_woningen) / 1000 )}

\CommentTok{# Time series object}
\NormalTok{vwts <-}\StringTok{ }\KeywordTok{ts}\NormalTok{(verkochte_woningen_per_kwartaal[}\StringTok{"totaal"}\NormalTok{], }
              \DataTypeTok{start =} \KeywordTok{c}\NormalTok{(}\DecValTok{1995}\NormalTok{, }\DecValTok{1}\NormalTok{), }\DataTypeTok{frequency =} \DecValTok{4}\NormalTok{)}

\NormalTok{v1 <-}\StringTok{ }
\StringTok{  }\NormalTok{vwts }\OperatorTok
\StringTok{  }\KeywordTok{autoplot}\NormalTok{() }\OperatorTok{+}
\StringTok{  }\KeywordTok{geom_smooth}\NormalTok{() }\OperatorTok{+}
\StringTok{  }\KeywordTok{labs}\NormalTok{(}
    \DataTypeTok{title =} \StringTok{"Transacties bestaande koopwoningen per kwartaal"}\NormalTok{,}
    \DataTypeTok{y =} \StringTok{"(x 1000)"}\NormalTok{,}
    \DataTypeTok{x =} \StringTok{''}
\NormalTok{    ) }

\NormalTok{v2 <-}\StringTok{ }
\StringTok{  }\NormalTok{vwts }\OperatorTok\StringTok{ }
\StringTok{  }\KeywordTok{ggsubseriesplot}\NormalTok{() }\OperatorTok{+}
\StringTok{  }\KeywordTok{labs}\NormalTok{(}
    \DataTypeTok{title =} \StringTok{"Seizoensplot"}\NormalTok{,}
    \DataTypeTok{y =} \StringTok{"(x 1000)"}\NormalTok{,}
    \DataTypeTok{x =} \StringTok{''}
\NormalTok{    ) }

\CommentTok{# Plot it}
\NormalTok{cowplot}\OperatorTok{::}\KeywordTok{plot_grid}\NormalTok{(v1, v2, }\DataTypeTok{ncol=}\DecValTok{1}\NormalTok{, }\DataTypeTok{rel_heights =} \KeywordTok{c}\NormalTok{(}\DecValTok{1}\NormalTok{, }\DecValTok{1}\NormalTok{))}
\end{Highlighting}
\end{Shaded}

\includegraphics{data_exploration_files/figure-latex/sales_arima-1.pdf}

\begin{Shaded}
\begin{Highlighting}[]
\CommentTok{# ACF plot}
\CommentTok{# Visualizes how much the most recent value of the series is correlated with past values of the series}
\NormalTok{o1 <-}\StringTok{ }\KeywordTok{ggAcf}\NormalTok{(vwts, }\DataTypeTok{lag.max =} \DecValTok{16}\NormalTok{) }\OperatorTok{+}
\StringTok{  }\KeywordTok{labs}\NormalTok{(}
    \DataTypeTok{title =} \StringTok{"ACF Plot seizoensinvloeden Verkopen Bestaande Koopwoningen"}\NormalTok{,}
    \DataTypeTok{y =} \StringTok{"Correlatie"}\NormalTok{,}
    \DataTypeTok{x =} \StringTok{"Aantal perioden lag"}
\NormalTok{    ) }

\CommentTok{# PACF Plot}
\CommentTok{# Visualizes whether certain lags are good for modeling or not; useful for data with a seasonal pattern}
\NormalTok{o2 <-}\StringTok{ }\KeywordTok{ggPacf}\NormalTok{(vwts, }\DataTypeTok{lag.max =} \DecValTok{16}\NormalTok{) }\OperatorTok{+}
\StringTok{  }\KeywordTok{labs}\NormalTok{(}
    \DataTypeTok{title =} \StringTok{"PACF Plot seizoensinvloeden Verkopen Bestaande Koopwoningen"}\NormalTok{,}
    \DataTypeTok{y =} \StringTok{"Correlatie"}\NormalTok{,}
    \DataTypeTok{x =} \StringTok{"Aantal perioden lag"}
\NormalTok{    ) }

\CommentTok{# Plot both}
\NormalTok{cowplot}\OperatorTok{::}\KeywordTok{plot_grid}\NormalTok{(o1, o2, }\DataTypeTok{ncol =} \DecValTok{1}\NormalTok{)}
\end{Highlighting}
\end{Shaded}

\includegraphics{data_exploration_files/figure-latex/sales_arima-2.pdf}

\begin{Shaded}
\begin{Highlighting}[]
\CommentTok{# Take first difference of the plot + seseonal effects}
\CommentTok{# Lag 1 year - lag 1 quarter - remainder fits model for Arima}
\NormalTok{vwts }\OperatorTok\StringTok{ }
\StringTok{  }\KeywordTok{diff}\NormalTok{(}\DataTypeTok{lag=}\DecValTok{4}\NormalTok{) }\OperatorTok\StringTok{ }
\StringTok{  }\KeywordTok{diff}\NormalTok{() }\OperatorTok\StringTok{ }
\StringTok{  }\KeywordTok{ggtsdisplay}\NormalTok{(}\DataTypeTok{main =} \StringTok{"Eerst orde verschil en seizoensinvloed"}\NormalTok{)}
\end{Highlighting}
\end{Shaded}

\includegraphics{data_exploration_files/figure-latex/sales_arima-3.pdf}

\begin{Shaded}
\begin{Highlighting}[]
\CommentTok{# Eventualy I got this model - which seems to fit beter then auto arima}
\NormalTok{fit <-}
\StringTok{  }\KeywordTok{Arima}\NormalTok{(vwts, }\DataTypeTok{order=}\KeywordTok{c}\NormalTok{(}\DecValTok{0}\NormalTok{,}\DecValTok{1}\NormalTok{,}\DecValTok{4}\NormalTok{), }\DataTypeTok{seasonal=}\KeywordTok{c}\NormalTok{(}\DecValTok{0}\NormalTok{,}\DecValTok{1}\NormalTok{,}\DecValTok{1}\NormalTok{))}
\KeywordTok{checkresiduals}\NormalTok{(fit)}
\end{Highlighting}
\end{Shaded}

\includegraphics{data_exploration_files/figure-latex/sales_arima-4.pdf}

\begin{verbatim}
## 
##  Ljung-Box test
## 
## data:  Residuals from ARIMA(0,1,4)(0,1,1)[4]
## Q* = 3.6063, df = 3, p-value = 0.3072
## 
## Model df: 5.   Total lags used: 8
\end{verbatim}

\begin{Shaded}
\begin{Highlighting}[]
\NormalTok{fit }\OperatorTok\StringTok{ }
\StringTok{  }\KeywordTok{forecast}\NormalTok{(}\DataTypeTok{h=}\DecValTok{12}\NormalTok{) }\OperatorTok\StringTok{ }
\StringTok{  }\KeywordTok{autoplot}\NormalTok{() }\OperatorTok{+}
\StringTok{  }\KeywordTok{labs}\NormalTok{(}
    \DataTypeTok{title =} \StringTok{"Voorspelling transactievolume per kwartaal koopwoningen"}\NormalTok{,}
    \DataTypeTok{y =} \StringTok{"(x 1000)"}\NormalTok{,}
    \DataTypeTok{x =} \StringTok{""}
\NormalTok{    ) }
\end{Highlighting}
\end{Shaded}

\includegraphics{data_exploration_files/figure-latex/sales_arima-5.pdf}

\hypertarget{conclusies}{%
\section{Conclusies}\label{conclusies}}

Er is in de media veel aandacht voor of de bouwplannen wel realiseerbaar
zijn. Het lijkt erop dat door regelgeving (PFAS, ruimtelijke ordening)
en problemen in de uitvoering de groei van het aantal huizen de groei
van de bevolking nog niet kan bijhouden waardoor het tekort op de
woningmarkt tijdelijk nog verder oploopt. Op termijn lijkt er meer
ruimte te ontstaan op de woningmarkt. De groei van de bevolking komt
momenteel bijna geheel vanuit de migratie. Het geboorteoverschot is
historisch laag - en zal gezien de vergrijzing ook niet snel stijgen.
Regionaal kan hierdoor worden verwacht dat de druk op de grote steden
verder zal toenemen. Dit is verder in deze studie niet onderzocht
vanwege gebrek aan data.

Statistisch gezien is in deze studie geen verband aangetoond tussen het
woningtekort en de oplopende huizenprijzen. Daarmee is geen invloed
aangetoond van een verschuiving in vraag en aanbod op de prijsvorming in
de huizenmarkt in Nederland.

De huizenprijzen reageren vertraagd op de ontwikkelingen in de economie.
Dit wordt zichtbaar in de leading indicators van het MARS model. Het
model lijkt een goede basis vormen om voorspellingen te kunnen maken
over de prijsontwikkelingen op de huizenmarkt. Hierbij moeten we wel
rekening houden met de huidige unieke omstandigheden van de economie
midden in de corona epidemie. Helaas is het in tijd die beschikbaar was
voor deze studie niet gelukt om ook daadwerkelijk voorspellingen te
maken voor de aankomende jaren op basis van dit model.

Het aantal transacties is in de financiële crisis van 2009 sterk
gekrompen. Sindsdien zien we een groeiversnelling, waar nog steeds geen
einde aan lijkt te komen. Het is in deze studie niet gelukt om een goed
model te vinden voor het voorspellen van het aantal transacties.
Opmerkelijk is dat ook het aantal huishoudens geen goede voorspelling
geeft van de transactievolumes. Op basis van de beschikbare data geeft
een autocorrelatie model (waarbij alleen gekeken worden naar het aantal
transacties in de voorgaande perioden) de beste voorspelling.

\hypertarget{referenties}{%
\section*{Referenties}\label{referenties}}
\addcontentsline{toc}{section}{Referenties}

\hypertarget{refs}{}
\leavevmode\hypertarget{ref-CBS2016}{}%
CBS. (2016). \emph{Indicatoren woningmarkt op groen}. Retrieved from
\url{https://www.cbs.nl/nl-nl/achtergrond/2016/16/indicatoren-woningmarkt-op-groen}

\leavevmode\hypertarget{ref-CBS2020a}{}%
CBS. (2020). \emph{Huizenmarkt in beeld}. Retrieved from
\url{https://www.cbs.nl/nl-nl/visualisaties/huizenmarkt-in-beeld}

\leavevmode\hypertarget{ref-R-timetk}{}%
Dancho, M., \& Vaughan, D. (2021). \emph{Timetk: A tool kit for working
with time series in r}. Retrieved from
\url{https://github.com/business-science/timetk}

\leavevmode\hypertarget{ref-Foreman2014}{}%
Foreman, J. W. (2014). \emph{Data Smart, Using Data Science to Transform
Information into Insight}. John Wiley \& sons, Inc.

\leavevmode\hypertarget{ref-Friedman2007}{}%
Friedman, J. H. (2007). Multivariate Adaptive Regression Splines.
\emph{Annals of Statistics}.
\url{https://doi.org/10.1214/aos/1176347963}

\leavevmode\hypertarget{ref-R-viridis}{}%
Garnier, S. (2018a). \emph{Viridis: Default color maps from matplotlib}.
Retrieved from \url{https://github.com/sjmgarnier/viridis}

\leavevmode\hypertarget{ref-R-viridisLite}{}%
Garnier, S. (2018b). \emph{ViridisLite: Default color maps from
matplotlib (lite version)}. Retrieved from
\url{https://github.com/sjmgarnier/viridisLite}

\leavevmode\hypertarget{ref-R-pdp}{}%
Greenwell, B. (2018). \emph{Pdp: Partial dependence plots}. Retrieved
from \url{https://CRAN.R-project.org/package=pdp}

\leavevmode\hypertarget{ref-R-vip}{}%
Greenwell, B., Boehmke, B., \& Gray, B. (2020). \emph{Vip: Variable
importance plots}. Retrieved from
\url{https://github.com/koalaverse/vip/}

\leavevmode\hypertarget{ref-pdp2017}{}%
Greenwell, B. M. (2017). Pdp: An r package for constructing partial
dependence plots. \emph{The R Journal}, \emph{9}(1), 421--436. Retrieved
from
\url{https://journal.r-project.org/archive/2017/RJ-2017-016/index.html}

\leavevmode\hypertarget{ref-vip2020}{}%
Greenwell, B. M., \& Boehmke, B. C. (2020). Variable importance
plots---an introduction to the vip package. \emph{The R Journal},
\emph{12}(1), 343--366. Retrieved from
\url{https://doi.org/10.32614/RJ-2020-013}

\leavevmode\hypertarget{ref-lubridate2011}{}%
Grolemund, G., \& Wickham, H. (2011). Dates and times made easy with
lubridate. \emph{Journal of Statistical Software}, \emph{40}(3), 1--25.
Retrieved from \url{https://www.jstatsoft.org/v40/i03/}

\leavevmode\hypertarget{ref-Groot2018}{}%
Groot, S., Vogt, B., Van der Wiel, K., \& Van Dijk, M. (2018).
\emph{Oververhitting op de Nederlandse huizenmarkt?} Retrieved from
\url{https://www.cpb.nl/publicatie/oververhitting-op-de-nederlandse-huizenmarkt/CPB-Achtergronddocument-1jun2018-Oververhitting-op-de-nederlandse-huizenmarkt.pdf}

\leavevmode\hypertarget{ref-Hanck2020}{}%
Hanck, C., Martin, A., Alexander, G., \& Schmelzer, M. (2020).
\emph{Introduction to Econometrics with R}. Retrieved from
\url{https://www.econometrics-with-r.org/index.html}

\leavevmode\hypertarget{ref-R-purrr}{}%
Henry, L., \& Wickham, H. (2020). \emph{Purrr: Functional programming
tools}. Retrieved from \url{https://CRAN.R-project.org/package=purrr}

\leavevmode\hypertarget{ref-R-stargazer}{}%
Hlavac, M. (2018). \emph{Stargazer: Well-formatted regression and
summary statistics tables}. Retrieved from
\url{https://CRAN.R-project.org/package=stargazer}

\leavevmode\hypertarget{ref-R-fma}{}%
Hyndman, R. (2020a). \emph{Fma: Data sets from "forecasting: Methods and
applications" by makridakis, wheelwright \& hyndman (1998)}. Retrieved
from \url{https://CRAN.R-project.org/package=fma}

\leavevmode\hypertarget{ref-R-fpp2}{}%
Hyndman, R. (2020b). \emph{Fpp2: Data for "forecasting: Principles and
practice" (2nd edition)}. Retrieved from
\url{https://CRAN.R-project.org/package=fpp2}

\leavevmode\hypertarget{ref-R-forecast}{}%
Hyndman, R., Athanasopoulos, G., Bergmeir, C., Caceres, G., Chhay, L.,
O'Hara-Wild, M., \ldots{} Yasmeen, F. (2021). \emph{Forecast:
Forecasting functions for time series and linear models}. Retrieved from
\url{https://CRAN.R-project.org/package=forecast}

\leavevmode\hypertarget{ref-R-expsmooth}{}%
Hyndman, R. J. (2015). \emph{Expsmooth: Data sets from "forecasting with
exponential smoothing"}. Retrieved from
\url{https://github.com/robjhyndman/expsmooth}

\leavevmode\hypertarget{ref-Hyndman2018}{}%
Hyndman, R. J., \& Athanasopoulos, G. (2018). \emph{Forecasting:
Principles and Practice}. Retrieved from \url{https://otexts.com/fpp2/}

\leavevmode\hypertarget{ref-forecast2008}{}%
Hyndman, R. J., \& Khandakar, Y. (2008). Automatic time series
forecasting: The forecast package for R. \emph{Journal of Statistical
Software}, \emph{26}(3), 1--22. Retrieved from
\url{https://www.jstatsoft.org/article/view/v027i03}

\leavevmode\hypertarget{ref-plotrix2006}{}%
J, L. (2006). Plotrix: A package in the red light district of r.
\emph{R-News}, \emph{6}(4), 8--12.

\leavevmode\hypertarget{ref-KennethGopalGerardvanLeeuwenDavidOmtzigt2020}{}%
Kenneth Gopal, Gerard van Leeuwen, David Omtzigt, T. K. en M. S.-F.
(2020). \emph{Prognose woningmarktmodel Socrates}. Retrieved from
\url{https://www.abfresearch.nl/publicaties/rapportage-socrates-2019/}

\leavevmode\hypertarget{ref-KennethGopalGerardvanLeeuwenDavidOmtzigt2019}{}%
Kenneth Gopal, Gerard van Leeuwen, David Omtzigt, T. K. en M., \&
Stuart-Fox. (2019). \emph{Socrates - Scenarioverkenningen van de
woningmarkt in 2030}. Retrieved from
\url{https://www.rijksoverheid.nl/documenten/rapporten/2019/05/29/socrates-2019---scenarioverkenningen-van-de-woningmarkt-in-2030}

\leavevmode\hypertarget{ref-R-caret}{}%
Kuhn, M. (2020). \emph{Caret: Classification and regression training}.
Retrieved from \url{https://github.com/topepo/caret/}

\leavevmode\hypertarget{ref-R-plotrix}{}%
Lemon, J., Bolker, B., Oom, S., Klein, E., Rowlingson, B., Wickham, H.,
\ldots{} Venables, B. (2021). \emph{Plotrix: Various plotting
functions}. Retrieved from
\url{https://CRAN.R-project.org/package=plotrix}

\leavevmode\hypertarget{ref-LeonGroenemeijerKennethGopal2020}{}%
Léon Groenemeijer, Kenneth Gopal, D. O. \&. G. van L. (2020).
\emph{Vooruitzichten bevolking, huishoudens en woningmarkt, Prognose en
Scenario's 2020-2035}. Retrieved from
\url{https://www.rijksoverheid.nl/documenten/rapporten/2020/06/12/vooruitzichten-bevolking-huishoudens-en-woningmarkt-prognose-en-scenarios-2020-2035}

\leavevmode\hypertarget{ref-R-sjPlot}{}%
Lüdecke, D. (2021). \emph{SjPlot: Data visualization for statistics in
social science}. Retrieved from
\url{https://strengejacke.github.io/sjPlot/}

\leavevmode\hypertarget{ref-OldfordR.J.MacKay2000}{}%
MacKay, R. J., \& Oldford, R. W. (2000). Scientific method, statistical
method and the speed of light. \emph{Statistical Science}, \emph{15}(3),
254--278. \url{https://doi.org/10.1214/ss/1009212817}

\leavevmode\hypertarget{ref-fma1998}{}%
Makridakis, S., Wheelwright, S., \& Hyndman, R. J. (1998).
\emph{Forecasting: Methods and applications}. John Wiley \& Sons.

\leavevmode\hypertarget{ref-R-plotmo}{}%
Milborrow, S. (2020). \emph{Plotmo: Plot a model's residuals, response,
and partial dependence plots}. Retrieved from
\url{http://www.milbo.users.sonic.net}

\leavevmode\hypertarget{ref-MinisterievanBinnenlandseZakenenKoninkrijksrelaties2021}{}%
Ministerie van Binnenlandse Zaken en Koninkrijksrelaties. (2021).
\emph{Datawonen}. Retrieved from
\url{https://datawonen.nl/dashboard/dashboard/koopprijs}

\leavevmode\hypertarget{ref-CBS2020b}{}%
Minister Wonen en Rijksdienst. (2016). \emph{Staat van de woningmarkt
2016}. Retrieved from
\url{https://www.rijksoverheid.nl/binaries/rijksoverheid/documenten/rapporten/2016/10/31/rapport-staat-van-de-woningmarkt-2016/rapport-staat-van-de-woningmarkt-2016.pdf}

\leavevmode\hypertarget{ref-CBS2020}{}%
Monitor koopwoningmarkt. (2020). \emph{CBS Achtergronddocument}.
Retrieved from
\url{https://www.cbs.nl/nl-nl/visualisaties/monitor-koopwoningmarkt}

\leavevmode\hypertarget{ref-R-here}{}%
Müller, K. (2020). \emph{Here: A simpler way to find your files}.
Retrieved from \url{https://CRAN.R-project.org/package=here}

\leavevmode\hypertarget{ref-R-tibble}{}%
Müller, K., \& Wickham, H. (2021). \emph{Tibble: Simple data frames}.
Retrieved from \url{https://CRAN.R-project.org/package=tibble}

\leavevmode\hypertarget{ref-sf2018}{}%
Pebesma, E. (2018). Simple Features for R: Standardized Support for
Spatial Vector Data. \emph{The R Journal}, \emph{10}(1), 439--446.
\url{https://doi.org/10.32614/RJ-2018-009}

\leavevmode\hypertarget{ref-R-sf}{}%
Pebesma, E. (2021). \emph{Sf: Simple features for r}. Retrieved from
\url{https://CRAN.R-project.org/package=sf}

\leavevmode\hypertarget{ref-urca2008}{}%
Pfaff, B. (2008). \emph{Analysis of integrated and cointegrated time
series with r} (Second). New York: Springer. Retrieved from
\url{http://www.pfaffikus.de}

\leavevmode\hypertarget{ref-R-urca}{}%
Pfaff, B. (2016). \emph{Urca: Unit root and cointegration tests for time
series data}. Retrieved from
\url{https://CRAN.R-project.org/package=urca}

\leavevmode\hypertarget{ref-R-base}{}%
R Core Team. (2021). \emph{R: A language and environment for statistical
computing}. Vienna, Austria: R Foundation for Statistical Computing.
Retrieved from \url{https://www.R-project.org/}

\leavevmode\hypertarget{ref-Research2020}{}%
Research, A. (2020). \emph{Primos - Bevolkingsprognose, Rapportage
Primos}. Retrieved from
\url{https://www.abfresearch.nl/producten/prognoses/primos-bevolkingsprognose/}

\leavevmode\hypertarget{ref-R-pacman}{}%
Rinker, T., \& Kurkiewicz, D. (2019). \emph{Pacman: Package management
tool}. Retrieved from \url{https://github.com/trinker/pacman}

\leavevmode\hypertarget{ref-pacman2018}{}%
Rinker, T. W., \& Kurkiewicz, D. (2018). \emph{pacman: Package
management for R}. Buffalo, New York. Retrieved from
\url{http://github.com/trinker/pacman}

\leavevmode\hypertarget{ref-R-broom}{}%
Robinson, D., Hayes, A., \& Couch, S. (2021). \emph{Broom: Convert
statistical objects into tidy tibbles}. Retrieved from
\url{https://CRAN.R-project.org/package=broom}

\leavevmode\hypertarget{ref-lattice2008}{}%
Sarkar, D. (2008). \emph{Lattice: Multivariate data visualization with
r}. New York: Springer. Retrieved from
\url{http://lmdvr.r-forge.r-project.org}

\leavevmode\hypertarget{ref-R-lattice}{}%
Sarkar, D. (2020). \emph{Lattice: Trellis graphics for r}. Retrieved
from \url{http://lattice.r-forge.r-project.org/}

\leavevmode\hypertarget{ref-R-rsample}{}%
Silge, J., Chow, F., Kuhn, M., \& Wickham, H. (2021). \emph{Rsample:
General resampling infrastructure}. Retrieved from
\url{https://CRAN.R-project.org/package=rsample}

\leavevmode\hypertarget{ref-R-TeachingDemos}{}%
Snow, G. (2020). \emph{TeachingDemos: Demonstrations for teaching and
learning}. Retrieved from
\url{https://CRAN.R-project.org/package=TeachingDemos}

\leavevmode\hypertarget{ref-R-lubridate}{}%
Spinu, V., Grolemund, G., \& Wickham, H. (2021). \emph{Lubridate: Make
dealing with dates a little easier}. Retrieved from
\url{https://CRAN.R-project.org/package=lubridate}

\leavevmode\hypertarget{ref-R-earth}{}%
Trevor Hastie, S. M. D. from mda:mars by, \& Thomas Lumley's leaps
wrapper., R. T. U. A. M. F. utilities with. (2020). \emph{Earth:
Multivariate adaptive regression splines}. Retrieved from
\url{http://www.milbo.users.sonic.net/earth/}

\leavevmode\hypertarget{ref-Vogt2018}{}%
Vogt, B., Kalara, N., \& Voogt, B. (2018). \emph{How do the Dutch
Finance their Own House?} Retrieved from
\url{https://www.cpb.nl/sites/default/files/omnidownload/CPB-Background-Document-June2018-How-do-the-dutch-finance-their-own-house.pdf}

\leavevmode\hypertarget{ref-R-corrplot}{}%
Wei, T., \& Simko, V. (2017a). \emph{Corrplot: Visualization of a
correlation matrix}. Retrieved from
\url{https://github.com/taiyun/corrplot}

\leavevmode\hypertarget{ref-corrplot2017}{}%
Wei, T., \& Simko, V. (2017b). \emph{R package "corrplot": Visualization
of a correlation matrix}. Retrieved from
\url{https://github.com/taiyun/corrplot}

\leavevmode\hypertarget{ref-reshape22007}{}%
Wickham, H. (2007). Reshaping data with the reshape package.
\emph{Journal of Statistical Software}, \emph{21}(12), 1--20. Retrieved
from \url{http://www.jstatsoft.org/v21/i12/}

\leavevmode\hypertarget{ref-ggplot22016}{}%
Wickham, H. (2016). \emph{Ggplot2: Elegant graphics for data analysis}.
Springer-Verlag New York. Retrieved from
\url{https://ggplot2.tidyverse.org}

\leavevmode\hypertarget{ref-R-stringr}{}%
Wickham, H. (2019a). \emph{Stringr: Simple, consistent wrappers for
common string operations}. Retrieved from
\url{https://CRAN.R-project.org/package=stringr}

\leavevmode\hypertarget{ref-R-tidyverse}{}%
Wickham, H. (2019b). \emph{Tidyverse: Easily install and load the
tidyverse}. Retrieved from
\url{https://CRAN.R-project.org/package=tidyverse}

\leavevmode\hypertarget{ref-R-reshape2}{}%
Wickham, H. (2020). \emph{Reshape2: Flexibly reshape data: A reboot of
the reshape package}. Retrieved from
\url{https://github.com/hadley/reshape}

\leavevmode\hypertarget{ref-R-forcats}{}%
Wickham, H. (2021a). \emph{Forcats: Tools for working with categorical
variables (factors)}. Retrieved from
\url{https://CRAN.R-project.org/package=forcats}

\leavevmode\hypertarget{ref-R-tidyr}{}%
Wickham, H. (2021b). \emph{Tidyr: Tidy messy data}. Retrieved from
\url{https://CRAN.R-project.org/package=tidyr}

\leavevmode\hypertarget{ref-tidyverse2019}{}%
Wickham, H., Averick, M., Bryan, J., Chang, W., McGowan, L. D.,
François, R., \ldots{} Yutani, H. (2019). Welcome to the tidyverse.
\emph{Journal of Open Source Software}, \emph{4}(43), 1686.
\url{https://doi.org/10.21105/joss.01686}

\leavevmode\hypertarget{ref-R-ggplot2}{}%
Wickham, H., Chang, W., Henry, L., Pedersen, T. L., Takahashi, K.,
Wilke, C., \ldots{} Dunnington, D. (2020). \emph{Ggplot2: Create elegant
data visualisations using the grammar of graphics}. Retrieved from
\url{https://CRAN.R-project.org/package=ggplot2}

\leavevmode\hypertarget{ref-R-dplyr}{}%
Wickham, H., François, R., Henry, L., \& Müller, K. (2021). \emph{Dplyr:
A grammar of data manipulation}. Retrieved from
\url{https://CRAN.R-project.org/package=dplyr}

\leavevmode\hypertarget{ref-R-readr}{}%
Wickham, H., \& Hester, J. (2020). \emph{Readr: Read rectangular text
data}. Retrieved from \url{https://CRAN.R-project.org/package=readr}

\leavevmode\hypertarget{ref-R-cowplot}{}%
Wilke, C. O. (2020). \emph{Cowplot: Streamlined plot theme and plot
annotations for ggplot2}. Retrieved from
\url{https://wilkelab.org/cowplot/}

\leavevmode\hypertarget{ref-knitr2014}{}%
Xie, Y. (2014). Knitr: A comprehensive tool for reproducible research in
R. In V. Stodden, F. Leisch, \& R. D. Peng (Eds.), \emph{Implementing
reproducible computational research}. Chapman; Hall/CRC. Retrieved from
\url{http://www.crcpress.com/product/isbn/9781466561595}

\leavevmode\hypertarget{ref-knitr2015}{}%
Xie, Y. (2015). \emph{Dynamic documents with R and knitr} (2nd ed.).
Boca Raton, Florida: Chapman; Hall/CRC. Retrieved from
\url{https://yihui.org/knitr/}

\leavevmode\hypertarget{ref-R-knitr}{}%
Xie, Y. (2021). \emph{Knitr: A general-purpose package for dynamic
report generation in r}. Retrieved from \url{https://yihui.org/knitr/}

\leavevmode\hypertarget{ref-R-dynlm}{}%
Zeileis, A. (2019). \emph{Dynlm: Dynamic linear regression}. Retrieved
from \url{https://CRAN.R-project.org/package=dynlm}

\leavevmode\hypertarget{ref-Formula2010}{}%
Zeileis, A., \& Croissant, Y. (2010). Extended model formulas in R:
Multiple parts and multiple responses. \emph{Journal of Statistical
Software}, \emph{34}(1), 1--13.
\url{https://doi.org/10.18637/jss.v034.i01}

\leavevmode\hypertarget{ref-R-Formula}{}%
Zeileis, A., \& Croissant, Y. (2020). \emph{Formula: Extended model
formulas}. Retrieved from
\url{https://CRAN.R-project.org/package=Formula}

\leavevmode\hypertarget{ref-zoo2005}{}%
Zeileis, A., \& Grothendieck, G. (2005). Zoo: S3 infrastructure for
regular and irregular time series. \emph{Journal of Statistical
Software}, \emph{14}(6), 1--27.
\url{https://doi.org/10.18637/jss.v014.i06}

\leavevmode\hypertarget{ref-R-zoo}{}%
Zeileis, A., Grothendieck, G., \& Ryan, J. A. (2021). \emph{Zoo: S3
infrastructure for regular and irregular time series (z's ordered
observations)}. Retrieved from \url{https://zoo.R-Forge.R-project.org/}

\leavevmode\hypertarget{ref-R-kableExtra}{}%
Zhu, H. (2021). \emph{KableExtra: Construct complex table with kable and
pipe syntax}. Retrieved from
\url{https://CRAN.R-project.org/package=kableExtra}

\end{document}
